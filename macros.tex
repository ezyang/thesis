%!TEX root = paper.tex
\usepackage[T1]{fontenc} % for tt curly braces
\usepackage{etoolbox} % for various programming stuff
\usepackage{amsmath}
\usepackage{listings}
\usepackage{array}
\usepackage{relsize}
\usepackage{amsthm}
\usepackage{amssymb}
\usepackage{color}
\usepackage[dvipsnames]{xcolor}
\usepackage{graphicx}
\usepackage{booktabs}
\usepackage{fancyvrb}
\usepackage{float}
\usepackage{stringstrings}
\usepackage{stmaryrd}
\usepackage{verbatimbox}
\usepackage{alltt}
\usepackage{multicol}
\usepackage[nooneline,bf]{subfigure}
% \usepackage[labelformat=simple]{subcaption}
% \renewcommand\thesubfigure{(\alph{subfigure})}
\usepackage{mdframed}
\usepackage{xcolor}
\usepackage{url}
\usepackage{xspace}
\usepackage[scaled]{beramono}

\lstset{basicstyle=\ttfamily,breaklines=true}

\newmdenv[
    hidealllines=true,
    backgroundcolor=black!20,
    skipbelow=\baselineskip,
    skipabove=\baselineskip
]{greybox}

\newcommand{\ie}{\emph{i.e.},\xspace}
\newcommand{\eg}{\emph{e.g.},~}
\newcommand{\etal}{\emph{et al.}}

\theoremstyle{definition}
\newtheorem{axiom}{Axiom}
\newtheorem{theorem}{Theorem}
\newtheorem{definition}{Definition}
\newtheorem{lemma}{Lemma}

\newcommand{\I}[1]{\ensuremath{\mathit{#1}}}
\newcommand{\cL}{{\calL}}
\newcommand{\Red}[1]{{\color{red} #1}}
\newcommand{\Blue}[1]{{\color{blue} #1}}
\newcommand{\Updated}[1]{\colorbox{red}{Updated! #1}}

% \newcommand{\Backpack}{\texttt{i}Backpack}%
\newcommand{\OldBackpack}{Backpack'14}%
\newcommand{\Backpack}{Backpack'16}%
\newcommand{\Ccomp}{Resolved component}%
\newcommand{\ccomp}{resolved component}%
\newcommand{\Unit}{Mixed component}%
\newcommand{\unit}{mixed component}%
\newcommand{\Iunit}{Instantiated component}%
\newcommand{\iunit}{instantiated component}%
\newcommand{\cid}{component identifier}%
\newcommand{\Cid}{Component identifier}%
\newcommand{\uid}{unit identifier}%
\newcommand{\Uid}{Unit identifier}%
\newcommand{\ir}{linked intermediate representation}
\newcommand{\Ir}{Linked intermediate representation}

% macros for unit syntax and semantic objects
% ------------------------------
% metavars
\newcommand{\Up}{p}
\newcommand{\UP}{P}
\newcommand{\Un}{n}
\newcommand{\UN}{N}
\newcommand{\UNs}{\mathit{Ns}}
\newcommand{\USn}{\mathit{Sn}}
\newcommand{\Ureqs}{\mathit{mreqs}}
\newcommand{\Uunit}{\mathit{mcomp}}
\newcommand{\mprog}{\mathit{mprog}}
\newcommand{\mcomp}{\mathit{mcomp}}
%\newcommand{\Uudecls}{\mathit{ldecls}}
\newcommand{\Uudecl}{\mathit{mdecl}}
\newcommand{\mdecl}{\mathit{mdecl}}
\newcommand{\mdecls}{\mathit{mdecls}}
\newcommand{\marg}{\mathit{marg}}
\newcommand{\Uhsexps}{\mathit{hsexps}}
\newcommand{\Uhsbody}{\mathit{hsbody}}
\newcommand{\Uhssig}{\mathit{hssig}}
\newcommand{\Uunitty}{\Xi}
\newcommand{\Umty}{T}
\newcommand{\Urole}{R}
\newcommand{\Uty}{\mathit{decl}}
\newcommand{\Utys}{\mathit{decls}}
\newcommand{\Udfun}{\texttt{\$} \I{fn}}
\newcommand{\Uinst}{\mathit{inst}}
\newcommand{\Uinsts}{\mathit{insts}}
\newcommand{\Ufinst}{\mathit{finst}}
\newcommand{\Ufinsts}{\mathit{finsts}}
\newcommand{\Umatch}{\mathit{sub}}
% syntax
\newcommand{\UsynUnitH}[2]{\ensuremath{\mathsf{component}~#1~#2}}
\newcommand{\UsynUnit}[3]{\ensuremath{\UsynUnitH{#1}{#2}~\{#3\}}}
\newcommand{\UsynLet}[2]{\ensuremath{\mathsf{let}~\modnameif{#1} = #2}}
\newcommand{\UsynLetMod}[3]{\ensuremath{\mathsf{let}~\modnameif{#1} = \Mod{#2}{#3}}}
\newcommand{\UsynMod}[2]{\ensuremath{\mathsf{module}~\modnameif{#1}~\{~#2~\}}}
\newcommand{\UsynSig}[2]{\ensuremath{\mathsf{signature}~\modnameif{#1}~\{~#2~\}}}
\newcommand{\UsynESig}[2]{\ensuremath{\mathsf{esignature}~\modnameif{#1}~\{#2\}}}
\newcommand{\UsynSigWith}[3]{\ensuremath{\mathsf{signature}~\modnameif{#1}~\{~#2~\}~\textsf{with}~#3}}
\newcommand{\UsynDep}[2]{\ensuremath{\mathsf{dependency}~#1~(#2)}}
\newcommand{\UsynDepEmpty}[1]{\ensuremath{\mathsf{dependency}~#1}}
% semantic objects
\newcommand{\UobjIface}{\mathsf{iface}}
\newcommand{\UobjTau}[2]{\ensuremath{\UobjIface\:(#1)~\{#2\}}}
\newcommand{\UobjUnitTy}[4]{\ensuremath{\forall~#1.\, \forall~#2 .\, #3 \rightarrow #4}}
\newcommand{\UobjUnitTyN}[3]{\ensuremath{\forall~#1 .\, #2 \rightarrow #3}}
% ------------------------------



\newcommand{\doubleplus}{\ensuremath{\mathbin{+\mkern-8mu+}}}
\newcommand{\scomp}[2]{\ensuremath{#2 \circ #1}}

\newcommand{\modname}[1]{\ensuremath{\mathtt{#1}}}%
\newcommand{\modnameif}[1]{%
  % make the argument a modname if it is not a macro (invocation)
  % and if it is capitalized
  \ifdefmacro{#1}%
    {#1}%
    {%
      \testcapitalized{#1}
      \ifcapitalized 
        \modname{#1}
      \else
        #1
      \fi%
    }%
  % % make the argument a modname if it is not a (primed) "m" or "n" metavar
  % \ifboolexpr{ test {\ifstrequal{#1}{m}} or %
  %              test {\ifstrequal{#1}{m'}} or %
  %              test {\ifstrequal{#1}{m''}} or %
  %              test {\ifstrequal{#1}{n}} or %
  %              test {\ifstrequal{#1}{n'}} or %
  %              test {\ifstrequal{#1}{n''}} }%
  %   {#1}
  %   {\modname{#1}}
}


\newcommand{\hole}[1]{\ensuremath{#1}}%
% \newcommand{\holevar}[1]{\ensuremath{\colorbox{Apricot}{\ensuremath{#1}}}}%
% \newcommand{\holevar}[1]{\ensuremath{\colorbox{Apricot}{\ensuremath{\modnameif{#1}}}}}
\newcommand{\holevar}[1]{\ensuremath{\langle\modnameif{#1}\rangle}}
\newcommand{\nholevar}[1]{\ensuremath{\{\modnameif{#1}\}}}
\newcommand{\hv}[1]{\holevar{#1}}
\newcommand{\nhv}[1]{\nholevar{#1}}

\newcommand{\cidl}[1]{\ensuremath{\mbox{\texttt{#1}}}}%
\newcommand{\cidlif}[1]{
  % make the argument a component ID if it is not a (primed) "p" or "q" metavar
  % NOTE: sometimes "p" is used as a literal! in such cases, use \cidl instead
  \ifdefmacro{#1}%
    {#1}%
    {%
      \ifboolexpr{ test {\ifstrequal{#1}{p}} or %
                   test {\ifstrequal{#1}{p'}} or %
                   test {\ifstrequal{#1}{p''}} or %
                   test {\ifstrequal{#1}{q}} or %
                   test {\ifstrequal{#1}{q'}} or %
                   test {\ifstrequal{#1}{q''}} }%
        {#1}
        {\cidl{#1}}%
    }
}
\newcommand{\icid}[2]{\cidlif{#1}[#2]}
\newcommand{\uidl}[2]{%
  \ensuremath{
    \cidl{#1}%
    [%
      % if subst is non-empty, allow line break after open paren
      \ifstrempty{#2}{}{\allowbreak}%
      #2
    ]%
  }}

% \newcommand{\Mod}[2]{\ensuremath{#1\!:\!#2}}%
\newcommand{\modcolon}{\hspace{.1ex}{:}\hspace{.15ex}}
\newcommand{\Mod}[2]{\ensuremath{#1\modcolon\modnameif{#2}}}%
\newcommand{\MOD}[3]{\ensuremath{\icid{\cidl{#1}}{#2}\modcolon\modnameif{#3}}}%
\newcommand{\subst}[2]{\ensuremath{\modnameif{#1} \mathop{=}\allowbreak #2}}
\newcommand{\substMod}[3]{\subst{#1}{\Mod{#2}{#3}}}
\newcommand{\substMOD}[4]{\subst{#1}{\MOD{#2}{#3}{#4}}}
\newcommand{\substHole}[1]{\subst{#1}{\holevar{#1}}}
\newcommand{\substw}[2]{#1\llbracket#2\rrbracket}
\newcommand{\substww}[3]{#1\llbracket#2\rrbracket\llbracket#3\rrbracket}
\newcommand{\prov}[2]{\ensuremath{\modnameif{#1} \,\mapsto\,\allowbreak #2}}
\newcommand{\provMod}[3]{\prov{#1}{\Mod{#2}{#3}}}
\newcommand{\provMOD}[4]{\prov{#1}{\MOD{#2}{#3}{#4}}}
\newcommand{\provHole}[1]{\prov{#1}{\holevar{#1}}}
\newcommand{\rename}[2]{\ensuremath{\modnameif{#1} \mapsto \modnameif{#2}}}


% now render unit IDs like
%   \uidl{mylib-3.0-aaa}{
%     \subst{m}{M},                                         % m and M are metavars
%     \substMod{Database}{\uidl{mysql-1.0-ccc}{}}{MySQL}    % Database, mysql..., MySQL are rendered
%    }


\newcommand{\haspr}{\:\triangleright\:}
\newcommand{\shapeis}{\triangleright}
%\newcommand{\lctx}{\mathcal{L}}
\newcommand{\lctx}{{\tilde{\Xi}}}
%\newcommand{\provs}{\mathcal{L}_\textsf{prov}}
%\newcommand{\provs}{{\tilde{\Sigma}}_\textsf{P}}
\newcommand{\provs}{{\tilde{\Sigma}}}
\newcommand{\impctx}{\mathcal{L}}
%\newcommand{\reqs}{{\tilde{\Sigma}}_\textsf{R}}
\newcommand{\reqs}{\Theta}
\newcommand{\pre}[1]{\tilde{#1}}
\newcommand{\preprovs}{\mathbb{P}}
\newcommand{\prereqs}{\mathbb{R}}
\newcommand{\preshape}{\mathbb{L}}
\newcommand{\Reqs}{\mathcal{R}}
\newcommand{\preP}{\preprovs}
\newcommand{\preR}{\prereqs}
\newcommand{\preL}{\preshape}
\newcommand{\shctx}{\tilde{\Gamma}}
%\newcommand{\lctxpair}[2]{\langle #1 \,;\, #2 \rangle}
\newcommand{\lctxpair}[2]{\forall #2.\, #1}
%\newcommand{\lctxpair}[2]{\lctxpairx{#1}{#2}}
\newcommand{\lctxpairx}[2]{\forall #2.\, \{ #1 \}}
%\newcommand{\lctxpairx}[2]{\begin{pmatrix}
%\textsf{requires:}& #2 \\
%\textsf{provides:}& #1 \\
%\end{pmatrix}}
\newcommand{\lctxpairex}[2]{\{#2\} &\rightarrow& \{#1\}}
\newcommand{\lift}[1]{\uparrow\!#1}
\newcommand{\shnull}{}
%\newcommand{\shnull}{\cdot}
%\newcommand{\shape}{\tilde{\Xi}}

% curly braces
% \newcommand{\lcb}{{\tt {\char 123}}}
% \newcommand{\rcb}{{\tt {\char 125}}}
\newcommand{\wrapcb}[1]{\texttt{\textbraceleft} #1 \texttt{\textbraceright}}

% S stands for "source", but these are not the true source declarations.

\newcommand{\Scomp}{\I{rcomp}}
\newcommand{\Sincl}{\I{rincl}}
\newcommand{\Sdecl}{\I{rdecl}}
\newcommand{\Scomponent}[2]{\texttt{component}~ #1 ~ \wrapcb{#2}}
\newcommand{\Sinclude}[2]{\texttt{mixin:}~ #1 ~#2}
\newcommand{\Sincludespec}[3]{\texttt{mixin:}~ #1 ~[\texttt{(}#2\texttt{)}]~[\texttt{requires}~ \texttt{(}#3\texttt{)}]}
\newcommand{\Slparen}{\texttt{(}}
\newcommand{\Srparen}{\texttt{)}}
\newcommand{\Sexposed}[2]{\texttt{exposed-module:}~ \modnameif{#1} ~\{ #2 \}}
\newcommand{\Sother}[2]{\texttt{other-module:}~ \modnameif{#1} ~\{ #2 \}}
\newcommand{\Srequired}[2]{\texttt{signature:}~ \modnameif{#1} ~\{ #2 \}}
\newcommand{\Sreexported}[2]{\texttt{reexported-module:}~ \modnameif{#1} ~\texttt{as}~ \modnameif{#2}}
\newcommand{\Sas}[2]{\modnameif{#1} \;\texttt{as}\; \modnameif{#2}}
\newcommand{\Scom}{\texttt{,}\,}

% macros for helper functions
\newcommand{\Freqs}[2][\Delta]{\ensuremath{\mathsf{reqnames}_{#1}(#2)}}
\newcommand{\Fpreshape}[1]{\ensuremath{\mathsf{preshape}(#1)}}
\newcommand{\FpreshapeI}[2][\Delta]{\ensuremath{\mathsf{preshape}_{#1}(#2)}}
\newcommand{\Fprelink}[1]{\ensuremath{\mathsf{merge}(#1)}}
\newcommand{\FprelinkTwo}[2]{\ensuremath{\mathsf{merge}(#1; #2)}}
\newcommand{\Flink}[1]{\ensuremath{\mathsf{link}(#1)}}
\newcommand{\Fsource}[1]{\ensuremath{\mathsf{source}(#1)}}
\newcommand{\Flet}[1]{\ensuremath{\mathsf{bind}(#1)}}
\newcommand{\Fprerename}[3]{\ensuremath{\mathsf{rnthin}(#1; #2; #3)}}
\newcommand{\Frename}[3]{\ensuremath{\mathsf{rnthin}(#1; #2; #3)}}
\newcommand{\Fprovs}[1]{\ensuremath{\mathsf{provs}(#1)}}
\newcommand{\Fdom}[1]{\ensuremath{\mathsf{dom}(#1)}}
\newcommand{\Fexports}[1]{\ensuremath{\mathsf{exps}(#1)}}
\newcommand{\Ftrim}[2]{\ensuremath{\mathsf{trim}_{#1}(#2)}}
\newcommand{\Ffnv}[1]{\ensuremath{\mathsf{fnv}(#1)}}

% notational convenience
\newcommand{\overlinei}[1]{\overline{#1}^i}
\newcommand{\overlinej}[1]{\overline{#1}^j}
\newcommand{\overlinek}[1]{\overline{#1}^k}
\newcommand{\overlinel}[1]{\overline{#1}^l}



\renewcommand{\topfraction}{.85}
\renewcommand{\bottomfraction}{.7}
\renewcommand{\textfraction}{.1}
\renewcommand{\floatpagefraction}{.9}
\renewcommand{\dbltopfraction}{.9}
\renewcommand{\dblfloatpagefraction}{.9}
\setcounter{topnumber}{9}
\setcounter{bottomnumber}{9}
\setcounter{totalnumber}{20}
\setcounter{dbltopnumber}{9}


\newcommand{\Edward}[1]{\textcolor{Red}{\bf[EZY: #1]}}
\newcommand{\Scott}[1]{\textcolor{Plum}{\bf[SK: #1]}}
\newcommand{\Derek}[1]{\textcolor{Blue}{\bf[DD: #1]}}
\newcommand{\Simon}[1]{\textcolor{RedOrange}{\bf[SPJ: #1]}}
\newcommand{\simon}[1]{\Simon{#1}}

% macros for examples

\newcommand{\uidArraysA}{%
  \uidl{arrays-a}{\holevar{Prelude}}%
}
\newcommand{\uidArraysB}{%
  \uidl{arrays-b}{\holevar{Prelude}}%
}
\newcommand{\uidStructures}{%
  \uidl{structures}{\holevar{Prelude}, \holevar{Array}}%
}
\newcommand{\uidStructuresA}{%
  \uidl{structures}{\holevar{Prelude},
                    \Mod{\uidArraysA}{Array}}%
}
\newcommand{\uidStructuresB}{%
  \uidl{structures}{\holevar{Prelude},
                    \Mod{\uidArraysB}{Array}}%
}

\newcommand{\Tmatch}[3]{#1 <:_{#2} #3}
\newcommand{\Tsubg}[3]{\{#1\} \quad #2 \quad \{ #3 \}}
\newcommand{\Tsub}[5]{\{#1\} \quad \Tmatch{#2}{#3}{#4} \quad \{ #5 \}}


\newcommand{\highlight}[1]{%
  \colorbox{yellow!50}{$\displaystyle#1$}}

\newcommand{\graybox}[1]{%
  \colorbox{gray!30}{$\displaystyle#1$}}

\newcommand{\KP}{\mathsf{P}}
\newcommand{\KM}{\mathsf{M}}

\newcommand{\RP}{\mathbb{P}}
\newcommand{\RM}{\mathbb{M}}
\newcommand{\RN}{\mathbb{N}}

\newcommand{\RT}{T}
\newcommand{\RE}{E}

\newcommand{\row}{\textsf{row}}
\newcommand{\rowty}[1]{\llparenthesis{}\,#1\,\rrparenthesis{}}
\newcommand{\record}[1]{\textsf{record}~#1}


\newcommand{\elab}[1]{\ensuremath{\leadsto{} #1}}


\newcommand{\CK}{\mathcal{K}}
\newcommand{\CT}{\mathcal{T}}

\newcommand{\Ictx}{\Gamma_c}
\newcommand{\IGamma}{\Gamma_c}
\newcommand{\IDelta}{\Delta_c}
\newcommand{\Idash}{\vdash_c}
\newcommand{\Impctx}{\impctx_c}
\newcommand{\iprog}{\I{iprog}}
\newcommand{\invariant}{\prec}

\newcommand{\tDelta}{\tilde{\Delta}}
\newcommand{\Quant}{\Theta}
\newcommand{\induces}{\Rightarrow}
% \newcommand{\declst}[3]{\begin{Bmatrix}#1\\#2\\#3\end{Bmatrix}}
\newcommand{\declst}[3]{\begin{Bmatrix}#2\\#3\end{Bmatrix}}
\newcommand{\Ideclst}[2]{\begin{Bmatrix}#1\\#2\end{Bmatrix}}
%\newcommand{\hsubis}{\rightarrow_S}
%\newcommand{\nhsubis}{\rightarrow_{\USn}}


\newcommand{\JMatch}[6]%
  [\Gamma; \Theta; \theta; \Delta; \Up_0]%
  {\ensuremath{#1 \vdash #2 : \forall #3.\, \forall #4.\, #5 \induces #6}}

\newcommand{\JReqType}[4]%
  % [\Gamma; \Up_0; \Theta; \theta; \Delta]%
  [\Gamma; \Theta; \theta; \Delta; \Up_0]%
  {\ensuremath{#1 \vdash #2 \,@\, #3 : #4}}

\newcommand{\thetaOR}{\hat\theta}
\newcommand{\SigmaOR}{\hat\Sigma}

% Alter some LaTeX defaults for better treatment of figures:
    % See p.105 of "TeX Unbound" for suggested values.
    % See pp. 199-200 of Lamport's "LaTeX" book for details.
    %   General parameters, for ALL pages:
    \renewcommand{\topfraction}{0.9}	% max fraction of floats at top
    \renewcommand{\bottomfraction}{0.8}	% max fraction of floats at bottom
    %   Parameters for TEXT pages (not float pages):
    \setcounter{topnumber}{2}
    \setcounter{bottomnumber}{2}
    \setcounter{totalnumber}{4}     % 2 may work better
    \setcounter{dbltopnumber}{2}    % for 2-column pages
    \renewcommand{\dbltopfraction}{0.9}	% fit big float above 2-col. text
    \renewcommand{\textfraction}{0.07}	% allow minimal text w. figs
    %   Parameters for FLOAT pages (not text pages):
    \renewcommand{\floatpagefraction}{0.7}	% require fuller float pages
	% N.B.: floatpagefraction MUST be less than topfraction !!
    \renewcommand{\dblfloatpagefraction}{0.7}	% require fuller float pages

	% remember to use [htp] or [htpb] for placement
        %
\lstset{language=Haskell,keywords={%  
    package, module, signature, link, unit, where, data, import, instance, hiding, type, dependency%
    }%
}%
\lstdefinelanguage{Cabal}
{
  % list of keywords
  morekeywords={
    name,version,library,exposed-modules,signatures,build-depends,main-is,executable,mixins
  },
  alsoletter=-,
  sensitive=false, % keywords are not case-sensitive
  morecomment=[l]{--}, % l is for line comment
  morestring=[b]" % defines that strings are enclosed in double quotes
}


\newcommand{\DIGatoms}{%
    \begin{array}{rcl}
    m & & \mbox{Module name} \\
    p, q & & \mbox{\Cid} \\
    \end{array}
}

\newcommand{\DIGsource}{%
    \begin{array}{rcl}
    \Uhsbody & & \mbox{Module source} \\
    \Uhssig & & \mbox{Signature source} \\
    \end{array}
}

\newcommand{\DIGresolved}{%
    \begin{array}{rcl}
    \Scomp & ::= & \Scomponent{\Up}{\overline{\Sdecl}} \\[0.3em]
    \Sdecl & ::= & \Sinclude{\Up}{\I{rns}} \\
           & |   & \Sexposed{m}{\Uhsbody} \\
           & |   & \Sother{m}{\Uhsbody} \\
           & |   & \Sreexported{m}{m'} \\
           & |   & \Srequired{m}{\Uhssig} \\
    \I{rns} & ::= & [\Slparen\overline{rn}\Srparen]~[\texttt{requires}~ \Slparen\overline{rn'}\Srparen] \\
    \I{rn} & ::= & \Sas{m}{m'} \\
           & |   & m \\
    \end{array}
}

\newcommand{\DIGuid}{%
    \begin{array}{rcll}
      M   &::=& \Mod{\UP}{m} & \text{Module identifier} \\
          &|&   \holevar{m} & \text{Module hole} \\
      \UP &::=& \icid{\Up}{S} & \text{\Uid} \\
      S   &::=& \overline{\subst{m}{M}} & \text{Module substitution} \\
    \end{array}
}

\newcommand{\DIGshape}{%
    \begin{array}{lcll}
    \lctx & ::= & \lctxpairx{\provs}{\reqs} & \mbox{Component shape} \\
    %r_\textsf{R} & ::= & \overline{m \rightarrow m'} & \mbox{Requirement renaming (total)} \\
    %r_\textsf{P} & ::= & \overline{m \rightarrow m'} & \mbox{Provision renaming (partial)} \\
    \reqs  & ::= & \overline{\hv{m}} & \mbox{Required module variables} \\
    \provs & ::= & \overline{\prov{m}{M}}  & \mbox{Provided modules} \\
    \shctx & ::= & \overline{\Up \shapeis \lctx} & \mbox{Component shape context} \\
    \end{array}
}

\newcommand{\DIGmixed}{%
    \begin{array}{rcl}
      \Uunit &::=& \UsynUnit{\Up}{\reqs}{\overline{\Uudecl}} \\
      \Uudecl &::=& \UsynDep{\UP}{r} \\
              &|&   \UsynMod{m}{\Uhsbody} \\
              &|&   \UsynSig{m}{\Uhssig} \\
              %& &   \qquad \textsf{with}~\overline{P\!:\!m} & \quad\text{(Merges)} \\
              %&|&   \UsynLet{m}{M} & \text{Let binding} \\
      %\mprog &::=& \overline{\Uunit} & \text{Mixed program} \\
      r   &::=& \overline{m \mapsto m'} \\
    \end{array}
}

\newcommand{\DIGinterface}{
\begin{array}{rcll}
  \Uunitty &::=& \forall \Theta.\, \forall \theta.\, \{\Sigma_R\} \rightarrow \{\Sigma_P\}
    & \text{Component type} \\
  \theta &::=& \overline{\hv{m.n}} & \text{Name variable quantifiers} \\
  \Sigma &::=& \overline{m : \Umty}
    & \text{Provided/required types} \\
  \Gamma &::=& \overline{\Up : \Uunitty} & \text{Component typing context} \\
  &&&\\
  \Umty &::=& \UobjTau{\UNs}{\overline{\Uty}~\overline{\Uinst}} & \text{Module type} \\
  % \tau_R &::=& \exists \overline{\hv{m.n}}.\, \tau  & \text{Signature type} \\
  \UNs &::=& \overline{\UN} & \text{Export specification} \\
  \Uty &::=& & \text{Defined entity spec} \\
       &  |& \texttt{data}~\graybox{n}~\overline{(a :: \kappa)} & \qquad\text{Abstract data declaration} \\
       &  |& \texttt{class}~\graybox{n}~\overline{(a :: \kappa)} & \qquad\text{Abstract type class} \\
       &  |& \texttt{type family}~\graybox{n} :: \kappa~\texttt{where}~\texttt{..} & \qquad\text{Abstract closed type family} \\
       &  |& \texttt{data}~\graybox{n}~\overline{(a :: \kappa)}~\texttt{where}~\I{dinfo}& \qquad\text{Data declaration} \\
       &  |& \texttt{newtype}~\graybox{n}~\overline{(a :: \kappa)} = \I{ntinfo}& \qquad\text{Newtype declaration} \\
       &  |& \texttt{type}~\graybox{n}~\overline{(a :: \kappa)} :: \kappa = \tau & \qquad\text{Type synonym} \\
       &  |& \texttt{class}~\graybox{n}~\overline{(a :: \kappa)}~\texttt{where}~\I{clinfo} & \qquad\text{Type class} \\
       &  |& \texttt{type family}~\graybox{n} :: \kappa~\texttt{where}~\I{tfinfo} & \qquad\text{Closed type family} \\
       &  |& \texttt{type family}~\graybox{n} & \qquad\text{Open type family} \\
       % Patterns omitted
       % Data families omitted
       &  |& n :: \tau& \qquad\text{Term declaration} \\
  % Notes: Why are the axioms/names kept separately from the instance
  % lists, rather than embedded "implicitly" (e.g., like how coercion
  % for newtype is stored)?  Historically, the DFun was always kept
  % separate from the instance declaration.
  \Uinst &::=& \texttt{instance}~n :: \tau & \text{Class instance} \\
         &  |& \texttt{family instance}~n :: \I{fiinfo} & \text{Family instance} \\
  &&&\\
  \Un   && \multicolumn{2}{l}{\text{Haskell source-level entity name}} \\
  \UN &::=& M.\Un & \text{Original name} \\
      &|&   \nhv{m.n} & \text{Name hole} \\
  \USn &::=& \overline{\subst{m.n}{N}} & \text{Name substitution} \\
  &&&\\
  a &&& \text{Haskell type variable} \\
  \tau &::=& \cdots N \cdots & \text{Haskell type} \\
  \kappa &::=& \cdots N \cdots & \text{Haskell kind} \\
  \multicolumn{3}{l}{\I{dinfo}, \I{ntinfo}, \I{clinfo}, \I{tfinfo}, \I{fiinfo}} & \text{Haskell declaration metadata} \\
% R &::=& & \text{Haskell role} \\
%   & | & \mathsf{P} & \qquad\text{Phantom} \\
%   & | & \mathsf{R} & \qquad\text{Representational} \\
%   & | & \mathsf{N} & \qquad\text{Nominal} \\
\end{array}
}


\DeclareMathOperator{\dom}{dom}
