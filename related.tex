%!TEX root = paper.tex
\chapter{Related Work}


\section{Comparison with \OldBackpack{}}

The inspiration for this
work was the original Backpack paper~\cite{backpack}.  Backpack was
first to pose the problem of retrofitting Haskell with interfaces, and
many of its design ideas, such as mixin packages, applicativity and
module identities have been preserved in this paper.  The primary contribution
of this paper is an actual \emph{implementation} of the
Backpack design, by refactoring of these ideas into a form that can be
implemented in two stages: mixin linking handled by the package manager,
and typechecking and compilation handled by the compiler.  However,
the vagaries of actual implementation have also influenced the design
and semantics of the language in some non-trivial ways, which we elaborate
upon below.

\paragraph{Unordered syntax}

\Backpack{}, unlike most existing module systems (including \OldBackpack{}
and most ML variants) has an \emph{unordered} module language.  Compare
to MixML, where the ordering of linking matters, even in the absence
of binding constructs:

\begin{figure}[H]
\begin{tabular}{p{0.45\textwidth} p{0.45\textwidth}}
\begin{lstlisting}[language=ML,escapechar=@]
(* REJECTED by paper MixML *)
{ type t,
  val f : t -> t,
  val x = f 2 }
with
{ type t = int }
\end{lstlisting}
&
\begin{lstlisting}[language=ML]
(* ACCEPTED by paper MixML *)
{ type t = int }
with
{ type t,
  val f : t -> t,
  val x = f 2 }
\end{lstlisting}
\end{tabular}
\caption{Order matters in MixML.  Note that mixml 0.2.1 accepts both programs, as it has a more general template pass which differs from the journal version of MixML.}
\label{fig:order-matters-in-mixml}
\end{figure}

In the first example, the first module is typechecked without knowledge
that \verb|type t = int|, and so the application \verb|f 2| fails to
typecheck.  In the second example, the results of bidirectional type
lookup are applied to \verb|t| before the module is typechecked,
causing us to accept the module.

When designing an unordered language, we must design our semantics so
that there is only one behavior which makes sense under any ordering.
One way to solve this problem is to precompute type information
(as in \OldBackpack{}) before typechecking proper.  This would resolve
the MixML issue above, since we would now be obligated to propagate
knowledge that \verb|type t = int| no matter where that declaration
occurred in the program.  However, as we've mentioned previously,
\Backpack{} does \emph{not} do a pre-pass in the interest of
separate compilation; this means there are a lot more choices available
to us.

Here are a few litmus tests:

\begin{lstlisting}
unit p where
    signature A where
        data T
        f :: T
unit q where
    signature B where
        f :: Int
unit r where
    dependency p[A=<A>]
    dependency q[A=<A>]
    signature A where
        type T = Int
\end{lstlisting}

Without the signature in \verb|r|, our matching judgment rejects the
merge of \verb|p| and \verb|q|'s signatures, since the abstract
type \verb|T| is not known to be equal to \verb|Int|.  However, if
we were to merge in \verb|type T = Int| before matching, the merge
would succeed.  Thus, \Backpack{} will typecheck the local signature
and merge it in, before checking matching.

The examples proliferate further when we consider a hypothetical,
recursive variant of \Backpack{}.  We suggested the principle that
type refinement only occur at the point when we are typechecking a
module; but should that also incorporate things from dependencies?

\begin{lstlisting}
unit t where
    module T where
        data T = MkT
unit p where
    signature A(T) where
        import T
    signature B(T) where
        import A
unit q where
    signature A(T) where
        data T
    signature C(T) where
        import B
unit r where
    dependency p[A=<A>,B=<B>]
    signature A(T) where
        import B
        import C
\end{lstlisting}

(By the way, this problem doesn't occur for \OldBackpack{} as it does
not permit reexports from signatures.)


\paragraph{Unordered Packages}

One of the most obvious surface language changes from \OldBackpack{}
to \Backpack{} is that \Backpack{}'s surface language is unordered: it
does not matter what order you write your signatures and modules, \Backpack{}
implicitly defines an ordering between them based on the import declarations
in the module text.  Unordered module languages are a better match for
the usual mapping of modules to files in a directory (which are unordered.)

This takes away some expressivity from \Backpack{}: for example, module
signatures (\verb|hsig| files) no longer subsume \verb|hs-boot| files
(GHC's current mechanism for supporting mutually recursive modules).
Within an unordered package, there is no way to tell if \verb|import A|
was intended to refer to a signature or a module.  (\verb|hs-boot| files
do not have this problem, as imports to the \verb|hs-boot| file are
explicitly disambiguated using a \verb|{-# SOURCE #-}| pragma).

\Red{Something about merging being undirected.  But maybe this makes sense
only for MixML}

\paragraph{Per-package modularity}

In \OldBackpack{}, type equality was based on a per-module computed
\emph{module identity}.  In effect, every defined module separately kept
track of the set of signatures that it transitively imported.
\Backpack{} posed the design constraint that mixin linking should not
inspect source code. Thus, \Backpack{} needs a coarser notion of
identity. In \Backpack{}, \emph{\uid{}s} record \emph{all} of the
requirements in the component, so that every module in a component
depends on the choice of implementation for every signature in the
component.  This can be inconvenient at times, as it means that if you
want a module to depend on fewer signatures, you must move it to a
separate package, but this restriction allows us to preserve the
abstraction barrier between the package manager and compiler.

\section{Comparison with MixML}

As \Backpack{} supports type synonyms where \OldBackpack{} did not,
it is instructive to do a detailed comparison of how \Backpack{} handles
many of the problems associated with transparent type declaration
which \OldBackpack{} sidesteps by only considering the \emph{names}
of language entities (rather than their computed types.)

It's worth recapping what has \emph{not} changed since \OldBackpack{}.
As before, \Backpack{} does not attempt to implement all of the features
of ML module systems.  We do not support first-class and higher order
units; indeed, from an implementation perspective, it is difficult to
see how these features would be implemented without compromising on
\Backpack{}'s promise that programming with signatures will be no less
efficient than programming against the implementing module directly.
Similarly, we do not support hierarchical linking or translucent
sealing.

\section{Modularity in the package manager}

While there is far more
literature on module systems that can be implemented entirely by a
compiler, there has been some work which has looked at the problem of
modular development at the package level.

One such system
is the Functoria DSL\footnote{\smaller\url{https://mirage.io/blog/introducing-functoria}}
of MirageOS~\cite{mirageos}.  MirageOS is a library operating system
written in OCaml, which provides modules and functors for constructing
unikernels.  Rather than manually instantiate these functors,
users write in the Functoria DSL, which describes what
dependencies to install (via the OCaml package manager) and how the ML
functors should be assembled.  Unlike \Backpack{}, their DSL follows
the model of explicit functor applications rather than mixin linking.

The Nix package manager~\cite{dolstra:thesis} is a
system for enabling reproducible builds of packages. Nix defines a (pure, functional) language of component
\emph{derivations}---\ie{} the source code and configuration
needed to build the derived component---%
functorized over configuration parameters and derivations of
depended-upon components.  Components are linked together with explicit
functor application, albeit with some of the syntactic convenience of
mixin linking.  However, there is no type system for components,
and thus the Nix output hashes (similar to our \cid{}s) only serve
the role of uniquely identifying derivations.

The SMLSC extension to Standard ML~\cite{swasey+:smlsc}, while
primarily intended as a mechanism to support separate compilation in
Standard ML, also has some similarities to \Backpack{}.  Like
Backpack, SMLSC operates at the level of \emph{units} (our
components), and defines interfaces between units to allow them to be
separately typechecked.  Unlike \Backpack{}, SMLSC does not support
reusing units with different
implementations of their interfaces: dependencies in SMLSC are always
\emph{definite references}, and signatures are used purely to permit
separate compilation.  In SMLSC, if you want multiple instantiations,
you are expected to use ML functors.

An unusual case of \emph{not} using a package manager when it would be
useful occurs in C++ templates.\footnote{\smaller%
  \url{https://gcc.gnu.org/onlinedocs/gcc/Template-Instantiation.html}}
C++ templates are
applicative, in the sense that two occurrences of \verb|vector<int>| refer to the
same type.  However, the C++ compiler must generate code when a
template is instantiated. Implemented naively, this could result in a
lot of duplicate copies of code.  One early method of handling this
problem, ``Cfront model'', involved a template database where
instances of templates were maintained.  However, it was too
complicated for most C++ compilers to handle this database, and so the
usual ``Borland model'' (implemented by GCC, among others) is to just
recompile every template instantiation and deduplicate them at link
time.  With Backpack, we already have a package manager, Cabal, which
administers its own installed package database, so we can offload the
caching of instantiated components to it.  (This technique would not
work for C++ templates, whose type based dispatch must be deeply
integrated with the compiler.)

\section{ML functors}

The original Backpack language distinguished
itself from ``functors'' in (variants of) the ML module system
\cite{milner+:def-of-sml-revised,ocaml} by the fact that it supported
separate type checking for recursive modules under applicative
instantiation.  Additionally, by being a mixin system, it is a
better fit for the package language and avoids the need for
sharing constraints.

As this paper does not address mutual recursion, one may wonder
if the \unit{} language is not simply just a stylized applicative
functor language. In fact, it is!  The reason our technical presentation
is done in the way it is done here is because our primary goal
was integrating with the existing compiler infrastructure.

\section{Mixin linking}

There is a rich literature in the mixin
linking world, which both Backpack and \Backpack{} draw heavily from
\cite{ancona+:cms,flatt+:units,rossberg+:mixml}.  Indeed,
the relationship to this literature is even clearer in \Backpack{}, as
the mixin linking step is factored out and is independent of the
Haskell language.  For example, the basic algorithm for linking in
Cardelli's \emph{linksets} calculus~\cite{cardelli:linksets}, at a
high level, is essentially the same algorithm as our mixin linking.
The difference, however, is that we must keep track of the structure
of the ``wiring diagram'', as this structure will be used to establish
the identities of types at the Haskell level.  In contrast, Cardelli
gave no account of the interaction between module-level linking and
core-level user-defined abstract data types.

The object oriented community has also studied mixin-style composition
in their designs.  However, these mechanisms are organized around
dynamic binding and objects; whereas in Backpack-style systems,
the emphasis is on packages.  Users of \Backpack{} pay no performance
penalty switching from a direct dependency to an indirect dependency
via a signature, because we don't do separate compilation of \Backpack{}.








%%% Local Variables:
%%% mode: latex
%%% TeX-master: "paper"
%%% End:
