\begin{figure}
\fbox{\textbf{Subroling:} $\rho \le \rho$}
\[
\textsf{N} \le \rho \qquad \rho \le \textsf{P} \qquad \rho \le \rho
\]
\caption{Subroling.}
\label{fig:subroling}
\end{figure}

\begin{figure}
\fbox{\textbf{Kinding:} $\Uty :: \kappa$}

\[
\begin{array}{lcll}
\texttt{data}~n~\overline{(a_i :: \kappa_i)} ~[\textsf{where}~ \I{dinfo}] &::& \overline{\kappa_i} \rightarrow \star & \textsc{(KData)}
\\
\texttt{newtype}~n~\overline{(a_i :: \kappa_i)} = \tau &::& \overline{\kappa_i} \rightarrow \star & \textsc{(KNewtype)}
\\
\texttt{type}~n~\overline{(a_i :: \kappa_i)} :: \kappa = \tau &::& \overline{\kappa_i} \rightarrow \kappa & \textsc{(KType)}
\\
\texttt{class}~n~\overline{(a_i :: \kappa_i)} ~[\textsf{where}~ \I{clinfo}] &::& \overline{\kappa_i} \rightarrow \textsf{Constraint} & \textsc{(KClass)}
\\
\texttt{type family}~n~\overline{(a_i :: \kappa_i)} :: \kappa ~[\texttt{where}~(\texttt{..} \,|\, \I{tfinfo})] &::& \overline{\kappa_i} \rightarrow \kappa & \textsc{(KTypeFam)}
\\
\texttt{data family}~n~\overline{(a_i :: \kappa_i)} :: \kappa &::& \overline{\kappa_i} \rightarrow \kappa & \textsc{(KDataFam)}
\\
n :: \tau &::& \star & \textsc{(KVal)}
\end{array}
\]
\caption{Defined entity kinding, where $\overline{\kappa_i} \rightarrow \kappa$ is interpreted as $\kappa_1 \rightarrow \kappa_2 \rightarrow \cdots \rightarrow \kappa$.}
\end{figure}

\begin{figure}
\fbox{\textbf{Roles:} $\ctx \vdash \textsf{roles}(\Uty) = \overline{\rho}$}
\[
\begin{array}{lcl}
\ctx \vdash \textsf{roles}( \texttt{data}~n~\overline{(a ::_\rho \kappa)}~[\texttt{where}~\I{dinfo}] )&=&\overline{\rho} \\
\ctx \vdash \textsf{roles}( \texttt{class}~n~\overline{(a ::_\rho \kappa)}~[\texttt{where}~\I{clinfo}] )&=&\overline{\rho} \\
\ctx \vdash \textsf{roles}( \texttt{type family}~n~\overline{(a :: \kappa)}~[\texttt{where}~(\texttt{..} \,|\, \I{tfinfo})]  )&=&\overline{\textsf{N}} \\
\ctx \vdash \textsf{roles}( \texttt{data family}~n~\overline{(a :: \kappa)}  )&=&\overline{\textsf{N}} \\
\ctx \vdash \textsf{roles}( \texttt{newtype}~n~\overline{(a ::_\rho \kappa)} = \I{ntinfo} )&=&\overline{\rho} \\
\ctx \vdash \textsf{roles}( n :: \tau )&=&\varnothing \\
\end{array}
\]
\[
\frac{
\ctx \vdash N : \Uty \qquad
\ctx \vdash \textsf{roles}(\Uty) = \overline{\rho'_i}~\overline{\rho'_j}
}{
\ctx \vdash \textsf{roles}( \texttt{type}~n~\overline{(a ::_\rho \kappa)} :: \kappa = N~\overline{\tau_i} ) = \overline{\rho}~\overline{\rho'_j}
}
\]
\caption{Roles of declarations.  Value declarations do not have roles.}
\end{figure}

\begin{figure}
\fbox{\textbf{Representational injectivity:} $\textsf{inj}(\Uty)$}
\[
\begin{array}{l}
\textsf{inj}( \texttt{data}~n~\overline{a}~\texttt{where}~\I{dinfo} ) \\
\textsf{inj}( \texttt{data family}~n~\overline{a} ) \\
\end{array}
\]
\caption{A type constructor is representationally injective if given $\texttt{T}~\overline{\tau_i} \sim_\textsf{R} \texttt{T}~\overline{\sigma_i}$, then $\overline{\tau_i \sim_{\rho_i} \sigma_i}$, where $\rho_i$ is the role of the $i$th type parameter.  This is only true for \emph{non-abstract} data types and data families.}
\label{fig:representational-injectivity}
\end{figure}
