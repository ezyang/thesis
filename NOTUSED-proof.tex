%!TEX root = paper.tex
\section{Pre-shaping}


\begin{figure}
\[
\begin{array}{@{}l}
  % pre syntax
  \begin{array}{rcll}
    \preshape   &::=& \lctxpair{\preprovs}{\prereqs}
      & \mbox{Component pre-shape} \\
    \preprovs  &::=& \overline{m}
      & \quad\mbox{Pre-provides set} \\
    \prereqs   &::=& \overline{m}
      & \quad\mbox{Pre-requires set} \\
    % \pre\Delta &::=& \overline{p \triangleright \preshape}
    %   & \mbox{Component pre-shape context}\\
  \end{array} \\
  \\
  

  % reqs
  \begin{array}{l}
    \Freqs{\Scomponent{p}{\overlinei{I_i}}{\overlinej{D_j}}} \;\;=\;\; \prereqs \\
    \qquad\text{where}\quad
      \left\{\begin{array}{c@{\;=\;}l}
        \overlinei{\preshape_i}  & \FpreshapeI{\overlinei{I_i}} \\
        \overlinej{\preshape'_j} & \Fpreshape{\overlinej{D_j}} \\
        \lctxpair{\_}{\prereqs}  & \Fprelink{\overlinei{\preshape_i}, \overlinej{\preshape'_j}} \\
      \end{array}\right. \\
  \end{array} \\
  \\

  % preshape(I)
  \begin{array}{l}
    \FpreshapeI{\Sinclude{p}{r_P}{r_R}} \\%{\texttt{include:}\: p\: (r_P)\: \texttt{requires}\: (r_R)} \\
    \qquad =\; \Fprerename{r_P}{\;\Fpreshape{\Delta(p)}}{\;r_R} \\
  \end{array} \\
  \\

  % preshape(L) ...
  \begin{array}{l@{\;\;=\;\;}l}
    \Fpreshape{\lctxpair{\provs}{\reqs}}
    & \lctxpair{\Fpreshape{\provs}}{\Fpreshape{\reqs}} \\
    \Fpreshape{\overline{m \mapsto M}} & \overline{m} \\
    \Fpreshape{\overline{m}} & \overline{m} \\
  \end{array} \\
  \\

  % preshape(D)
  \begin{array}{l@{\quad=\quad}l}
    \Fpreshape{\Sexposed{m}{\_}}%{\texttt{exposed-module:}~ m = \_}
    & \lctxpair{m}{\varnothing} \\
    \Fpreshape{\Sother{m}{\_}}%{\texttt{other-module:}~ m = \_}
    & \lctxpair{m}{\varnothing} \\
    \Fpreshape{\Srequired{m}{\_}}%{\texttt{required-signature:}~ m = \_}
    & \lctxpair{\varnothing}{m} \\
    \Fpreshape{\Sreexported{\_}{\_}}%{\texttt{reexported-module:}~ \_ ~\texttt{as}~ \_}
    & \lctxpair{\varnothing}{\varnothing} \\
  \end{array} \\
  \\

  % merge(Ls) and rename(r,L,r)
  \begin{array}{l@{\;\;=\;\;}l}
    \Fprelink{\overlinei{\lctxpair{\preP_i}{\preR_i}}}
    & \left\langle
        \left(\biguplus_i \preP_i\right)
        ~;\,
        \left(\bigcup_i \preR_i\right)
          -
          \left(\biguplus_i \preP_i\right)
      \right\rangle \\
    % \FprelinkTwo{\lctxpair{\preprovs}{\prereqs}}{\lctxpair{\preprovs'}{\prereqs'}}
    % & \lctxpair{\preprovs \uplus \preprovs'}{(\prereqs \cup \prereqs') - (\preprovs \uplus \preprovs')} \\
    \Fprerename{r_P}{\lctxpair{\preprovs}{\prereqs}}{r_R}
    & \lctxpair{r_P(\preprovs)}{r_R(\prereqs)} \\
  \end{array} \\

\end{array}
\]

\caption{\Red{Todo: This needs to be updated for syntax changes.} Algorithmic determination of component requirements via pre-shaping.
$r(\preP)$ denotes the image of the renaming $r$ treated as a function
on the set $\preP$.}
\label{fig:pre-shaping}
\end{figure}


\section{Proofs}

\section{Definitions}

First, we define the language of instantiated components which can be
compiled.  An instantiated component is simply a \unit{} equipped
with a closed substution:
%
\[
  \I{icomp} ::= \UsynUnit{p}{S}{\overline{\mdecl}} \qquad \text{Instantiated component} \\
\]
%
Compilation takes place in a different context than typechecking:
while in typechecking the \emph{component typing context} records
the uninstantiated types for components, the \emph{compilation typing context}
records only fully instantiated components, which have no free
module or name variables.
%
\[
  \Ictx ::= \overline{P : \Sigma} \qquad \text{Compilation typing context}
\]
%
An $\Ictx$ is well-formed if every $P$ mentioned within it is in $\Ictx$.
We maintain the following relationship between $\Gamma$ and $\Ictx$:

\begin{definition}
A compilation typing context $\Ictx$ is a concretization of a
typing context $\Gamma$, if for all $p[S] : \Sigma \in \Ictx$
and any choice of $\Delta$ and $p_0 \neq p$, it is the case that
$\Gamma; \cdot; \Delta; \Up_0 \vdash p[S] : \Sigma$; i.e., every instantiated component type in $\Ictx$ must be consistent with the type computed in context $\Gamma$.
\end{definition}
%
The local context $\Delta$ and import context $\impctx$ remain the same, but we will distinguish the substituted variants that occur during compilation with
subscripts: $\IDelta, \Impctx$.  These will have the following
relationship:

\begin{definition}
A local context $\IDelta = \Sigma_R; \Sigma_P$ is a concretization of
$\Delta$ under substitution
$S$ in context $\Gamma$ for component $p_0$, if
given $\Gamma; \cdot; (\Sigma_R; \Sigma_P); p_0 \vdash S : \Sigma_R \induces \USn$
then $\IDelta = \substww{\Delta}{S}{\USn}$
\end{definition}

\begin{definition}
An import context $\Impctx$ is a concretization of $\impctx$ under substitution
$S$ if $\Impctx = \substw{\impctx}{S}$.
\end{definition}

Now, we must define the judgments for compilation against the compilation
typing context.  Some of these judgments are identical to those for
typechecking, modulo $\Gamma$ versus $\Gamma_c$ and the omission
of $\Theta$ (we are always closed); the main differences
are highlighted in gray. \\

\fbox{$\Ictx \Idash \UP : \Sigma$}
\begin{greybox}
\[
\frac{
\UP : \Sigma \in \Ictx
}{
\Ictx \Idash \UP : \Sigma
}
\]
\end{greybox}

\fbox{$\Ictx; \IDelta; P_0 \Idash M : \tau$}
\[
\frac{
P \neq P_0 \qquad
\Ictx \Idash P : \Sigma \qquad
m : \tau \in \Sigma
}{
\Ictx; \IDelta; P_0 \Idash \Mod{P}{m} : \tau
}
\]
\[
\frac{
m : \tau \in \IDelta
}{
\Ictx; \IDelta; P_0 \Idash \Mod{P_0}{m} : \tau
}
\]
\\

\fbox{$\Ictx; \IDelta; P \Idash \UN :: \Uty$}
\[
\frac{
\Ictx; \IDelta; P \Idash M : \tau \qquad
n :: \Uty \in \tau
}{
\Ictx; \IDelta; P \Idash M.n :: \Uty
}
\]

As before, we assume appropriate Haskell judgments: \\

\fbox{$\Ictx; \IDelta; \Impctx; M \Idash \Uhsbody : \tau$}

\begin{greybox}
\fbox{$\Ictx; \IDelta; \Impctx; P_0; M \Idash \Uhssig : \tau$}
\end{greybox}

Notably, the judgment for a signature takes in a full module identity
$M$ identifying what the implementing module is.  To compile a
signature, we simply look up $\Ictx; \IDelta; P_0 \Idash M : \tau$ and
thin the export list according to the declarations in the signature.

Declaration rules: \\

%   \begin{theorem}
%   If $\Gamma; \Delta; P_0 \vdash P : \Sigma$ and $\Ictx$ is a concretization
%   of $\Gamma$, then $\Ictx; \Delta; P_0 \Idash P : \Sigma$.
%   \end{theorem}
%   \begin{proof}
%   \end{proof}

\fbox{$\shctx; \Ictx; P_0 \Idash \Ideclst{\IDelta}{\Impctx} \quad \overline{\Uudecl} \quad \Ideclst{\IDelta'}{\Impctx'}$}
\[
\frac{
\Ictx; \IDelta; \Impctx; \Mod{P_0}{m} \Idash \Uhsbody : \tau
}{
\shctx; \Ictx; P_0 \Idash \Ideclst{\Sigma_R; \Sigma_P}{\Impctx} \UsynMod{m}{\Uhsbody} \Ideclst{\Sigma_R; (\Sigma_P, m : \tau)}{\Impctx}
}
\]
\begin{greybox}
\[%
\frac{
\IGamma; \IDelta; \Impctx; p_0[S]; S(m) \Idash \Uhssig : \tau
}{
\shctx; \IGamma; p_0[S] \Idash \Ideclst{\Sigma_R; \Sigma_P}{\Impctx} \UsynSig{m}{\Uhssig} \Ideclst{(\Sigma_R, m : \tau); \Sigma_P}{\Impctx}
}
\]
\[
\frac{
\begin{array}{c}
p \haspr \lctx \in \shctx \qquad
\lctxpair{\provs}{\reqs} = \substw{\substw{\lctx}{S}}{S_0} \qquad
r = \overline{m \mapsto m'} \\
\reqs \subseteq \textsf{dom}(\IDelta) \qquad
\IGamma \vdash \substw{p[S]}{S_0} : \Sigma
\end{array}
}{
\shctx; \IGamma; p_0[S_0] \Idash \Ideclst{\IDelta}{\Impctx} \UsynDep{p[S]}{r} \Ideclst{\IDelta}{\Impctx,  \overline{m \mapsto \provs(m')}}
}
\]
\end{greybox}

In fact, the rule for handling dependencies is a generalization
of the type-checking rule: in the case of type checking $S_0$ is
always the identity substitution.\\

\fbox{$\shctx; \IGamma \Idash \Uunit : \Sigma$}
\[
\frac{
\shctx; \IGamma; p[S] \Idash \{ \cdot, \cdot \} \quad \overline{\Uudecl} \quad \{ (\Sigma_R; \Sigma_P) ; \Impctx \}
}{
\begin{array}{l}
\shctx; \IGamma \Idash \UsynUnit{p}{S}{\overline{\Uudecl}} : \Sigma_P
\end{array}
}
\]\\

\fbox{$\shctx; \IGamma \Idash \mprog$}
\[
\frac{
\begin{array}{c}
\shctx; \IGamma \Idash \UsynUnit{p}{S}{\overline{\Uudecl}} : \Sigma \\
\shctx; \IGamma, (p[S] : \Sigma) \Idash \mprog \\
\end{array}
}{
\shctx; \IGamma \Idash \UsynUnit{p}{S}{\overline{\Uudecl}}; \mprog
}
\]
\[
\shctx; \IGamma \Idash \cdot
\]

\section{Theorem}

We now prove the invariance of typing under substitution.

\begin{definition}
We define the invariant $\Gamma; \IGamma; p \vdash \Delta, \impctx \invariant_S \IDelta, \Impctx$ as:
\begin{enumerate}
\item $\IGamma$ is a concretization of $\Gamma$,
\item $\IDelta$ is a concretization of $\Delta$ under substitution $S$ in context $\Gamma$ for package $p$,
\item $\Impctx$ is a concretization of $\impctx$ under substitution $S$, and
\item (Completeness.) For every $m \mapsto \Mod{P}{m'} \in \impctx$, $\substw{\UP}{S}$ is in $\IGamma$.
\end{enumerate}
\end{definition}

\begin{axiom}[Invariance of Haskell module typing under substitution]
Given $\Delta = \Sigma_R; \Sigma_P$, and:
\begin{enumerate}
\item $\Gamma; \Theta; \Delta; \impctx; p_0; m \vdash \Uhsbody : \tau$,
\item $\Gamma; \cdot; \Delta; p_0 \vdash S : \Sigma_R \induces \USn$, and
\item The invariant $\Gamma; \IGamma; p_0 \vdash \Delta, \impctx \invariant_S \IDelta, \Impctx$ holds,
\end{enumerate}
Then $\IGamma; \IDelta; \Impctx; p_0[S] \Idash \Uhsbody : \substww{\tau}{S}{\USn}$.
\end{axiom}

\begin{axiom}[Invariance of Haskell signature typing under substitution]
Given $\Delta = \Sigma_R; \Sigma_P$, and:
\begin{enumerate}
\item $\Gamma; \Theta; \Delta; \impctx; p_0; m \vdash \Uhssig : \tau$,
\item $\Gamma; \cdot; \Delta; p_0 \vdash S : \Sigma_R  \induces \USn$, and
\item The invariant $\Gamma; \IGamma; p_0 \vdash \Delta, \impctx \invariant_S \IDelta, \Impctx$ holds,
\end{enumerate}
Then $\IGamma; \IDelta; \Impctx; p_0[S]; S(m) \Idash \Uhssig : \substww{\tau}{S}{\USn}$.
\end{axiom}

\begin{theorem}[Invariance of declaration typing under substitution]
\label{thm:mdecl}
Given an $\mdecl$ and a substitution $S$, with $\Delta = \Sigma_R; \Sigma_P$ given:
\begin{enumerate}
\item $\shctx; \Gamma; p_0 \vdash \{ \Theta; \Delta; \impctx \} ~ \mdecl ~ \{ \Theta'; \Delta'; \impctx' \}$,
\item $\Gamma; \cdot; \Delta; p_0 \vdash S : \Sigma_R \induces \USn$,
\item The invariant $\Gamma; \IGamma; p_0 \vdash \Delta, \impctx \invariant_S \IDelta, \Impctx$ holds,
\item When $\mdecl = \UsynDep{P}{r}$, then $\substw{P}{S}$ is in $\IGamma$, and
\item If $p[S] \in \IGamma$, with $p : \lctx \in \shctx$, for every $m \mapsto \Mod{P}{m'} \in \substw{\lctx}{S}$, $P$ is in $\IGamma$.
\end{enumerate}
Then:
\begin{itemize}
\item $\shctx; \IGamma; p_0[S] \Idash \{ \IDelta; \Impctx \} ~ \mdecl ~ \{ \IDelta'; \Impctx' \}$, and
\item The invariant is preserved: $\Gamma; \IGamma \vdash \Delta', \impctx' \invariant_S \IDelta', \Impctx'$.
\end{itemize}
\end{theorem}
\begin{proof}
By case analysis on (1):
\begin{itemize}
\item $\UsynMod{m}{\Uhsbody}$:
    \begin{itemize}
        \item Progress $\shctx; \IGamma; p[S] \vdash \{ \IDelta; \Impctx \} ~ \mdecl ~ \{ \IDelta'; \Impctx' \}$:
        \begin{itemize}
          \item Show $\Ictx; \IDelta; \Impctx; \Mod{p[S]}{m} \Idash \Uhsbody : \tau$ using axiom:
          \begin{itemize}
            \item $\Gamma; \Theta; \Delta; \impctx; p_0; m \vdash \Uhsbody : \tau$ by (1).
            \item $\Gamma; \cdot; \Delta; p_0 \vdash S : \Sigma_R \induces \USn$ by (2).
            \item The invariant holds by (3).
          \end{itemize}

        \end{itemize}
        \item Preservation: add $m : \substww{\tau}{S}{\USn}$ (from axiom) to $\Sigma_P$, by distributivity of substitution.
    \end{itemize}
    \item $\UsynSig{m}{\Uhssig}$: similar to module case.
    \item $\UsynDep{P}{r}$:
    \begin{itemize}
        \item Progress: immediate, premise $\IGamma \vdash \substw{P}{S} : \Sigma$
            holds by (4).
        \item Preservation: add $\overline{m_i \mapsto \substw{\provs}{S}(m'_i)}$ to $\Impctx$.  Concretization is immediate; completeness holds by (5).
    \end{itemize}
\end{itemize}
\end{proof}

\begin{theorem}[Invariance of component typing under substitution]
Given an $\mcomp = \UsynUnit{p}{\overline{m}}{\overline{mdecl}}$ and a substitution $S$, given:
\begin{enumerate}
\item $\shctx; \Gamma \vdash \mcomp : \Xi$,
\item $\Gamma, p : \Xi \vdash p[S] : \Sigma$,
\item $\IGamma$ is a concretization of $\Gamma$,
\item For every $\UsynDep{P}{r}$, $\substw{P}{S}$ is in $\IGamma$, and
\item If $p[S] \in \IGamma$, with $p : \lctx \in \shctx$, for every $m \mapsto \Mod{P}{m'} \in \substw{\lctx}{S}$, $P$ is in $\IGamma$.
\end{enumerate}
Then $\shctx; \IGamma \Idash \UsynUnit{p}{S}{\overline{mdecl}} : \Sigma$
\end{theorem}
\begin{proof}
Proof by induction on (1), repeatedly applying
Theorem~\ref{thm:mdecl} and observing the invariant is preserved.
\end{proof}

%   \begin{theorem}
%   For an $\mprog$ which instantiates to an $\iprog$, if $\shctx; \cdot \vdash \mprog$,
%   then $\shctx; \cdot \vdash \iprog$.
%   \end{theorem}
%   \begin{proof}
%   The inductive structure here is tricky, but if we do induction on the instantiation
%   tree the desired result should hold.
%   \end{proof}
