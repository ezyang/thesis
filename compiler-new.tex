%!TEX root = paper.tex
\chapter{Type checking}
\label{sec:compiler}

We now give a formal model for how our implementation typechecks
\unit{}s which still have unfilled requirements.  It's not possible to
give a full semantics, since that would involve formalizing all of
Haskell as implemented by GHC, but we will informally explain the
operation of the judgments we assume are given to us.

\section{Semantic objects}

\begin{figure}
\[ \DIGinterface{} \]
\caption{Semantic objects of GHC Haskell with \Backpack{}}
\label{fig:semantic-objects}
\end{figure}

The semantic objects of GHC Haskell, extended with \Backpack{},
are given in Figure~\ref{fig:semantic-objects}.
Unlike source code, identifiers in these objects have been renamed into \emph{original names} ($N$) which identify the exact module which originally defined an entity.  Every \I{decl} is associated with an entity name $\graybox{n}$, which we have indicated in grey.  We leave some of the details about declaration metadata unspecified (e.g., \I{dinfo}, \I{ntinfo}, etc.), as they are not directly related to Backpack.  Abstract declarations can only occur in signatures.

A module type consists of four components:

\begin{enumerate}
    \item The export specification ($\UNs$), which specifies what entities are brought
    into scope when a module is imported.
    \item The defined entity specifications ($\Utys$), which give definitions
    for each entity provided by a module.  The defined entity specification for
    an original name $M.n$ is recorded in the module type for $M$.
    \item A set of instances ($\Uinsts$), which specifies the type class
    instances defined in this module.
    \item The set of modules transitively imported by this module ($\Uimps$),
    which controls the set of instances which are in scope when performing
    type class resolution.
\end{enumerate}

We collect module types into a context $\Phi$, associating a module type
with an uninstantiated module identity.

\paragraph{Differences from GHC 8.2's representation}  The account of semantic objects
given here closely corresponds to GHC's \emph{interface file representation}; however
there are some major differences.

\begin{itemize}
    \item In our formulation of GHC Haskell's semantic objects, an
    instance directly records the type representing the constraint that
    the instance fills.  In GHC Haskell, an instance instead contains a
    reference to a \emph{dictionary function}, which is just a type of
    term declaration that is never exported.  The two presentations are
    equivalent, but embedding the dictionary function in the instance
    makes it easier to specify the rules instance checking.  Similarly,
    closed family instances are associated with a coercion axiom, but in
    our presentation they are embedded in \I{tfinfo}.

    \item Export specification is just a list of names, rather than names with
    children; the latter is used to support import declarations like
    \verb|Maybe(..)|.

    \item GHC doesn't record the set of modules transitively imported (which would
    mean that interface files take quadratic space to store); instead
    it records only those modules which define orphan instances for family instances.
\end{itemize}

\section{Auxiliary definitions}

Projection out of interfaces

\fbox{$T ~\mathsf{exports}~ \UNs$}

\[ \UobjTau{\UNs}{\Utys\, \Uinsts\, \Uimps} ~\mathsf{exports}~ \UNs \]

Projection out of $N$

\fbox{$N ~\mathsf{names}~ n$}

\[ \nhv{m.n} ~\mathsf{names}~ n \]
\[ M.n ~\mathsf{names}~ n \]

Interpreting export list $\UNs$ as mapping on $n$. ($n$ is deterministic as $\UNs$ is non-overlapping)

\[ \frac{
N \in \UNs \qquad
N ~\mathsf{names}~ n
}{
\UNs(n) = N
} \]

Lifting export list $\UNs$ to name substitution $\USn$

\[
\frac{
\forall i.\, N_i ~\mathsf{names}~ n_i
}{
\mathsf{nsubst}(m, \overline{N_i}) = \overline{\nhv{m.n_i} \mapsto N_i}
}
\]

\subsection{Typing rules}


% Key points:
%   - Type lookup is always done by finding it in the context.
%     Only situation this doesn't happen is primitive typechecking of
%     module or signature.  Important, because when we check
%     compatibility, type lookup might be done with regards to
%     something strange.  (Things get even stranger with
%     recursion; then we preemptively put the types in the
%     context.)
%   - Signatures process one-by-one, as an earlier signature
%     can cause a later one to merge.  Same with exports.
% Properties we want: (1) signature can import other signature, (2) signature sees the same view of an import A as other modules would, (3) semantics should be forwards compatible with a recursive semantics, (4) signature should be able to refine type so that merging that would otherwise fail succeeds
%   - Non-recursive case is easier because we don't have to preemptively
%     merge all types (running afowl of (4))
%   - Thinning doesn't work too well: want to thin BEFORE you typecheck
%   so that you avoid typechecking things you don't know.
%   - Thinning has a relationship with explicit signature exports


Context: $\Gamma ::= \Phi; \Delta; P_0$

EPS is $\Phi ::= \overline{\Mod{p}{m} : T}$. HPT is $\Delta ::= \overline{m : T}$.

$P_0$ is ``self''.  Invariant: it is always fully uninstantiated or fully instantiated. (Does it work with partial instantiation?)


\textsf{rename} is functorial on interface.  Key case is handling name.

\fbox{$\Gamma \vdash \textsf{rename}_S~ N \leadsto N$}

\begin{itemize}

\item
Ordinary names just have substitution applied to them.

\[
\Gamma \vdash \textsf{rename}_S~ M.n \leadsto M[S].n
\]

\item
A hole name must be substituted by consulting the exports
of the module substituted in.  (This assumes that the export
list is sufficient, which it will be if we checked if the
substitution matches.)

\[
\frac{
\Gamma \vdash S(m) : T \qquad
T ~\textsf{exports}~ \UNs
}{
\Gamma \vdash \textsf{rename}_S~ \nhv{m.n} \leadsto \UNs(n)
}
\]

\end{itemize}

% TODO: Need to also update transitively reachable modules when this
% occurs

Typing module identities

\fbox{$\Gamma \vdash M : T$}

\[
\frac{
m : T \in \Delta
}{
\Phi; \Delta; P_0 \vdash \hv{m} : T
}
\]

\[
\frac{
m : T \in \Delta
}{
\Phi; \Delta; P_0 \vdash \Mod{P_0}{m} : T
}
\]

\[
\frac{
\Mod{p}{m} : T \in \Phi \qquad
\Phi; \Delta; P_0 \vdash \textsf{rename}_S~ T \leadsto T'
}{
\Phi; \Delta; P_0 \vdash \Mod{p[S]}{m} : T'
}
\]


Get the types of all signatures

\fbox{$\Gamma \vdash M ~\textsf{requires}~ T$}

\[
\frac{
\begin{array}{c}
% Look up each signature
\Mod{p}{m} : T \in \Gamma \\
% And do the provisional renaming (without assuming m in Delta)
\Gamma \vdash \textsf{renamesig}_{m,S}~ T \leadsto T' \\
\end{array}
}{
\Gamma \vdash \Mod{p[S]}{m} ~\textsf{requires}~ T' \\
}
\]


\begin{figure}

\fbox{\textbf{Top-level typing:} $\Phi \vdash \mcomp : \Delta$}

\[
\frac{
1
}{
\Phi \vdash \UsynUnit{p}{\Theta}{\mdecls}
}
\]

\fbox{\textbf{Declaration typing:} $\Gamma; \mathcal{L}; \mathcal{D} \vdash \Uudecl : T$}

\[
\frac{
\Gamma; \mathcal{L}; m \vdash \Uhsbody : T \\
}{
\Gamma; \mathcal{L}; \mathcal{D} \vdash \UsynMod{m}{\Uhsbody} : T
}
\]

\[
\frac{
\begin{array}{c}
% Typecheck the local signature
\Phi; \Delta; P_0; \mathcal{L}; m \vdash \Uhssig : T_0 \\
% Retrieve the signatures to merge in
\forall M_i \in \mathcal{D}(m).\quad
    \Phi; \Delta; P_0 \vdash M_i ~\textsf{requires}~ T_i \\
% Get the exports
\forall i.\quad T_i ~\textsf{exports}~ \UNs_i \\
% Shape and apply it
\forall i.\quad T'_i = \substw{T_i}{\textsf{nsubst}(m, \sideset{}{_m}\bigoplus\limits_i \overline{\UNs_i})} \\
% Pre-merge the signatures
\bigoplus\limits_i T'_i = T_M \\
% Check compatibility, with merged context
\forall i.\quad \Phi; \Delta, m : T_M; P_0 \vdash \hv{m} \le T'_i \\
\end{array}
}{
\Phi; \Delta; P_0; \mathcal{L}; \mathcal{D} \vdash \UsynSig{m}{\Uhssig} : T_M
}
\]

\caption{Typing rules}
\end{figure}



Renaming signatures which are not in $\Delta$ (in $\Gamma$).  Functorial on types.  Otherwise defers to usual renaming.

\fbox{$\Gamma \vdash \textsf{renamesig}_{m,S}~ T \leadsto T$}
\fbox{$\Gamma \vdash \textsf{renamesig}_{m,S}~ N \leadsto N$}

\[
\frac{
S(m) = \hv{m'}
}{
\Gamma \vdash \textsf{renamesig}_{m,S}~ \nhv{m.n} \leadsto \nhv{m'.n} \\
}
\]

\[
\frac{
N ~\textsf{names}~ n \qquad
N \neq \nhv{m.n} \qquad
\Gamma \vdash \textsf{rename}_S~ N \leadsto N' \
}{
\Gamma \vdash \textsf{renamesig}_{m,S}~ N \leadsto N' \\
}
\]

\begin{figure}

\fbox{\textbf{Instance resolution:} $\Gamma \vdash \Uimps ~\textsf{solves}~ \Uinst$}

\fbox{\textbf{Type equality:} $\Gamma \vdash \tau =_\textsf{hs} \tau$}

\fbox{\textbf{Kind equality:} $\Gamma \vdash \kappa =_\textsf{hs} \kappa$}

\fbox{\textbf{Signature typechecking:} $\Gamma; \mathcal{L}; m \vdash \I{hssig} : T$}

\fbox{\textbf{Module typechecking:} $\Gamma; \mathcal{L}; m \vdash \I{hsmod} : T$}

\caption{Assumed Haskell judgments}

\end{figure}

\begin{figure}
\fbox{\textbf{Signature type merging:} $\ctx; m \vdash \bigoplus_i T_i \leadsto T$}

\[
\frac{
\begin{array}{c}
% Shape and apply it
\UNs' = \bigoplus_{m,i} \textsf{exports}(T_i) \qquad
\USn' = \textsf{nsubst}(m, \UNs')\qquad
\overline{\I{decls}_i} ~\textsf{acyclic}\\
%\forall i.\quad \substw{T_i}{} = \UobjTau{\UNs'_i}{\Utys'_i;\, \Uinsts'_i;\, \Uimps'_i} \\
T_P = \UobjTau{\UNs'}{\bigoplus_i \substw{\Utys_i}{\USn'}} \\
\Gamma; \Delta, m : T^-_P; P_0 \vdash \bigoplus_i \substw{\Uinsts_i}{\USn'} \leadsto \Uinsts'\\
T_M = \UobjTau{\UNs'}{\bigoplus_i \substw{\Utys_i}{\USn'}; \Uinsts'; \bigcup_i \Uimps_i} \\
\forall i.\quad \Gamma; \Delta, m : T^-_M; P_0 \vdash \hv{m} \le T_i
\end{array}
}{
\begin{array}{c}
\ctx; m \vdash \bigoplus_i\, T_i \leadsto T_M
\end{array}
}
\quad(\textsc{MSigType})
\]

\fbox{\textbf{Export list merging:} $\UNs \oplus_m \UNs = \UNs \qquad N \oplus_m N = N$}

\[
\frac{
\begin{array}{c}
\forall j.\quad \textsf{occname}(N_j) = \textsf{occname}(N'_j) \qquad
\textsf{occname}(\overline{N_i}) \,\#\, \textsf{occname}(\overline{N'_k}) \\
\end{array}
}{
\overline{N_i}, \overline{N_j} \oplus_m \overline{N'_j}, \overline{N'_k} = \overline{N_i}, \overline{N_j \oplus_m N'_j}, \overline{N'_k}
}
\quad(\textsc{MExport})
\]

\[
N \oplus_m \nhv{m.n} = N
\qquad\qquad
\nhv{m.n} \oplus_m N = N
\qquad\qquad
N \oplus_m N = N
\quad(\textsc{MName})
\]

\fbox{\textbf{Instance merging:} $\ctx \vdash \bigoplus_i \Uinsts_i \leadsto \Uinsts$}
\[
\frac{
\Uinsts = \bigcup_i \Uinsts_i \mathclose{}/\mathopen{} \ctx \vdash - =_\textsf{hs} -
}{
\ctx \vdash \bigoplus_i \Uinsts_i \leadsto \Uinsts
}
\]

\fbox{\textbf{Declaration list merging:} $\Utys \oplus \Utys = \Utys \qquad \Uty \oplus \Uty = \Uty$}
\[
\frac{
\forall j.~ \textsf{occname}(\Uty_j) = \textsf{occname}(\Uty'_j) \qquad \textsf{occname}(\overline{\Uty_i}) \,\#\, \textsf{occname}(\overline{\Uty'_k})
}{
\overline{\Uty_i}, \overline{\Uty_j} \oplus \overline{\Uty'_j}, \overline{\Uty'_k} = \overline{\Uty_i}, \overline{\Uty_j \oplus \Uty'_j}, \overline{\Uty'_k}
}
\quad(\textsc{MDecls})
\]

\[
\begin{array}{c}
\texttt{data}~n~\overline{(a ::_\rho \kappa)} \oplus \texttt{data}~n~\overline{(a ::_{\rho'} \kappa')} = \texttt{data}~n~\overline{(a ::_{\rho \oplus \rho'} \kappa)} \\
\texttt{class}~n~\overline{(a ::_\rho \kappa)} \oplus \texttt{class}~n~\overline{(a ::_{\rho'} \kappa')} = \texttt{class}~n~\overline{(a ::_{\rho \oplus \rho'} \kappa)} \\
\Uty \oplus \Uty' = \Uty \quad \textrm{if}~\Uty'~\textrm{abstract}\\
\Uty \oplus \Uty' = \Uty' \quad \textrm{otherwise}\\
\end{array}
\]


\fbox{\textbf{Role merging:} $\rho \oplus \rho = \rho$}
\[
\textsf{P} \oplus \rho = \textsf{P} \qquad
\rho \oplus \textsf{P} = \textsf{P} \qquad
\textsf{N} \oplus \rho = \rho \qquad
\rho \oplus \textsf{N} = \rho
\]

\caption{Merging rules.  $\Uty \oplus \Uty$ rules are top-to-bottom, apply the first one that matches.}
\label{typing:merging}
\end{figure}

\begin{figure}
\fbox{\textbf{Kinding:} $\Uty :: \kappa$}

\[
\begin{array}{lcl}
\texttt{data}~n~\overline{(a_i :: \kappa_i)} ~[\textsf{where}~ \I{dinfo}] &::& \overline{\kappa_i} \rightarrow \star
\\
\texttt{newtype}~n~\overline{(a_i :: \kappa_i)} = \tau &::& \overline{\kappa_i} \rightarrow \star
\\
\texttt{type}~n~\overline{(a_i :: \kappa_i)} :: \kappa = \tau &::& \overline{\kappa_i} \rightarrow \kappa
\\
\texttt{class}~n~\overline{(a_i :: \kappa_i)} ~[\textsf{where}~ \I{clinfo}] &::& \overline{\kappa_i} \rightarrow \textsf{Constraint}
\\
\texttt{type family}~n~\overline{(a_i :: \kappa_i)} :: \kappa ~[\texttt{where ..} \,|\, \I{tfinfo}] &::& \overline{\kappa_i} \rightarrow \kappa
\\
\texttt{type family}~n~\overline{(a_i :: \kappa_i)} &::& \overline{\kappa_i} \rightarrow \kappa
\\
n :: \tau &::& \star
\end{array}
\]
\caption{Defined entity kinding, where $\overline{\kappa_i} \rightarrow \kappa$ is interpreted as $\kappa_1 \rightarrow \kappa_2 \rightarrow \cdots \rightarrow \kappa$.  Note that no context $\Gamma$ is needed as the kind is fully encoded in declaration.}
\end{figure}

\begin{figure}

\fbox{\textbf{Module subtyping:} $\ctx \vdash T \le T$}

\[
\frac{
\begin{array}{c}
T_M = \UobjTau{\UNs'}{\Utys'~\Uinsts'~\Uimps'} \\
\UNs' \supseteq \UNs \qquad
\Uimps' \supseteq \Uimps \\
\forall \I{decl} \in \I{decls}.\quad \Gamma; \Delta, m : T_M^-; P_0 \vdash \UNs'(\textsf{occname}(\Uty)) \le \Uty \\
\forall \I{inst} \in \I{insts}.\quad \Gamma; \Delta, m : T_M^-; P_0 \vdash \Uimps' ~\textsf{solves}~ \I{inst}
\end{array}
}{
%\ctx \vdash \UobjTau{\overline{N_i}\, \overline{N_j}}{\Utys\, \Uinsts} \le \UobjTau{\overline{N_i}}{\overline{\Uty'}\, \overline{\Uinst'}}
\ctx \vdash T_M \le \UobjTau{\UNs}{\Utys\, \Uinsts\, \Uimps}
}
\quad(\textsc{SubMod})
\]

\caption{
Every declaration is rooted by an export.  Any non-exported declarations are dropped when typechecking signature; if they are thinned out that is an error.
}
\label{typing:top-subtyping}


\end{figure}


\begin{figure}

\fbox{\textbf{Declaration subtyping:} $\ctx \vdash \Uty \le \Uty$}

\[
\frac{
\begin{array}{c}
\ctx \vdash N : \Uty \qquad
\Uty :: \kappa \qquad
\Uty' :: \kappa' \\
\ctx \vdash \kappa =_\mathsf{hs} \kappa' \qquad
\ctx \vdash \Uty \le_\textsf{pre} \Uty'
\end{array}
}{
\ctx \vdash N \le \Uty'
}
\quad(\textsc{SubDecl})
\]

\fbox{\textbf{Declaration pre-subtyping:} $\ctx \vdash \Uty \le_\textsf{pre} \Uty$}

\[
\ctx \vdash \texttt{data}~n~\overline{(a_i :: \kappa_i)} \le_\textsf{pre} \texttt{data}~n~\overline{(a_i :: \kappa'_i)}
\quad(\textsc{SubAbsData})
\]
\[
\ctx \vdash \texttt{class}~n~\overline{(a_i :: \kappa_i)} \le_\textsf{pre} \texttt{class}~n~\overline{(a_i :: \kappa'_i)}
\quad(\textsc{SubAbsClass})
\]
\[
\begin{array}{c}
\ctx \vdash \texttt{type family}~n~\overline{(a_i :: \kappa_i)} :: \kappa ~\texttt{where ..} \\
\qquad\qquad\le_\textsf{pre} \texttt{type family}~n~\overline{(a_i :: \kappa'_i)} :: \kappa' ~\texttt{where ..}
\end{array}
\quad(\textsc{SubAbsClosedTF})
\]
\[
\ctx \vdash \texttt{type family}~n~\overline{(a_i :: \kappa_i)} \le_\textsf{pre} \texttt{type family}~n~\overline{(a_i :: \kappa'_i)}
\quad(\textsc{SubOpenTF})
\]

% Only slightly non-trivial pre-subtyping relations

\[
\frac{
\ctx;\, \overline{a_i :: \kappa_i} \vdash \I{dinfo} =_\textsf{hs} \I{dinfo}'
}{
\ctx \vdash \texttt{data}~n~\overline{(a_i :: \kappa_i)} ~\texttt{where}~ \I{dinfo} \le_\textsf{pre} \texttt{data}~n~\overline{(a_i :: \kappa'_i)} ~\texttt{where}~ \I{dinfo}'
}
\quad(\textsc{SubData})
\]

\[
\frac{
\ctx;\, \overline{a_i :: \kappa_i} \vdash \I{ntinfo} =_\textsf{hs} \I{ntinfo}'
}{
\ctx \vdash \texttt{newtype}~n~\overline{(a_i :: \kappa_i)} = \I{ntinfo} \le_\textsf{pre} \texttt{newtype}~n~\overline{(a_i :: \kappa'_i)} = \I{ntinfo}'
}
\quad(\textsc{SubNewtype})
\]

\[
\frac{
\ctx;\, \overline{a_i :: \kappa_i} \vdash \tau =_\textsf{hs} \tau'
}{
\ctx \vdash \texttt{type}~n~\overline{(a_i :: \kappa_i)} = \tau \le_\textsf{pre} \texttt{type}~n~\overline{(a_i :: \kappa'_i)} = \tau'
}
\quad(\textsc{SubType})
\]

\[
\frac{
\ctx;\, \overline{a_i :: \kappa_i} \vdash \I{clinfo} =_\textsf{hs} \I{clinfo}'
}{
\ctx \vdash \texttt{class}~n~\overline{(a_i :: \kappa_i)} ~\texttt{where}~ \I{clinfo} \le_\textsf{pre} \texttt{class}~n~\overline{(a_i :: \kappa'_i)} ~\texttt{where}~ \I{clinfo}'
}
\quad(\textsc{SubClass})
\]

\[
\frac{
\ctx;\, \overline{a_i :: \kappa_i} \vdash \I{tfinfo} =_\textsf{hs} \I{tfinfo}'
}{
\begin{array}{l}
\ctx \vdash \texttt{type family}~n~\overline{(a_i :: \kappa_i)} ~\texttt{where}~ \I{tfinfo}
\\ \qquad \le_\textsf{pre} \texttt{type family}~n~\overline{(a_i :: \kappa'_i)} ~\texttt{where}~ \I{tfinfo}'
\end{array}
}
\quad(\textsc{SubClosedTF})
\]

\[
\frac{
\ctx \vdash \tau =_\textsf{hs} \tau'
}{
\ctx \vdash n :: \tau \le_\textsf{pre} n :: \tau'
}
\quad(\textsc{SubValue})
\]

% Nontrivial pre-subtyping relations

\begin{mdframed}
\[
\ctx \vdash \texttt{data}~n~\overline{(a_i :: \kappa_i)} ~\texttt{where}~ \I{dinfo} \le_\textsf{pre} \texttt{data}~n~\overline{(a_i :: \kappa'_i)}
\quad(\textsc{SubDataAbsData})
\]
\[
\ctx \vdash \texttt{newtype}~n~\overline{(a_i :: \kappa_i)} :: \kappa = \tau \le_\textsf{pre} \texttt{data}~n~\overline{(a_i :: \kappa'_i)}
\quad(\textsc{SubNewtypeAbsData})
\]
% Nullary type synonym!
\[
\frac{
\ctx \vdash \tau ~\textsf{has no type family applications}
}{
\ctx \vdash \texttt{type}~n :: \kappa = \tau \le_\textsf{pre} \texttt{data}~n~\overline{a_i :: \kappa'_i}
}
\quad(\textsc{SubTypeAbsData})
\]
\[
\ctx \vdash \texttt{class}~n~\overline{(a_i :: \kappa_i)}~\texttt{where}~ \I{clinfo} \le_\textsf{pre} \texttt{class}~n~\overline{(a_i :: \kappa'_i)}
\quad(\textsc{SubClassAbsClass})
\]
\[
\ctx \vdash \texttt{type}~n :: \kappa = \tau \le_\textsf{pre} \texttt{class}~n~\overline{(a_i :: \kappa'_i)}
\quad(\textsc{SubTypeAbsClass})
\]
\[
\begin{array}{c}
\ctx \vdash \texttt{type family}~n~\overline{(a_i :: \kappa_i)} :: \kappa ~\texttt{where}~ \I{tfinfo} \\
\le_\textsf{pre} \texttt{type family}~n~\overline{(a_i :: \kappa'_i)} :: \kappa' ~\texttt{where ..}
\end{array}
\quad(\textsc{SubClosedTFAbsClosedTF})
\]
\end{mdframed}
\caption{Subtyping relations.}
\label{typing:subtyping}
%   For concision, this relation is factored into
%   subtyping relation, and a pre-subtyping relation which doesn't consider kind
%   equality.  The boxed relations are the nontrivial subtyping relations, while
%   the rest are reflexive up to kind and type equivalence in Haskell.
\end{figure}


Collect holes.  Given $p[S]$, for each $\hv{m}$, emit $p[S]$ requires $m$; and for each $\Mod{P'}{m'}$ otherwise, recursively handle $P'$.

% Things to talk about:
%   - Temporary bogus types. MixML and Backpack have this problem
%     (Backpack in particular: if you define a type and refer to it,
%     but that type is already in context, which do you use.  Name
%     is consistent, but not necessarily definition!)
%   - Cyclic exports when you add signature reexports to recursive
%   Backpack
%   - Recursive design space is big. hs-boot for hsigs? Do signature
%   declarations become before or after dependencies?  Signature
%   should see anything, but also refine recursive occurences of self?
