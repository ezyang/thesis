\appendix

\chapter{The Double Vision Problem}
\label{sec:double-vision}

First coined by Dreyer in his thesis~\cite{dreyer:thesis}, the
\emph{double vision problem} refers to technical difficulty that arises
when attempting to define the typing rules for recursive modules.
Informally, the double vision problem occurs when the things an implementing
module knows about a type does not coincide with the things they know about the
recursive occurrence of that very same type.

There is no formal definition of ``double vision'', which at times makes it
difficult to characterize, because module systems can solve double vision
to varying degrees.  It will be
helpful to explain the double vision problem as a series of litmus tests,
which will help characterize the various facets of this problem.

\section{Simple recursion}

\paragraph{Simple recursion in OCaml}  Our first litmus test is a simple
recursive module from Rossberg\footnote{\url{http://caml.inria.fr/pub/ml-archives/caml-list/2007/05/d9414d45a9a6f30f2609e08c43f011a0.en.html}}, which fails to compile in OCaml 4.02:

% chktex-file 13
% chktex-file 26
% chktex-file 36
\begin{figure}[H]
\begin{tabular}{p{0.35\textwidth} p{0.55\textwidth}}
\begin{lstlisting}[language=ML,escapechar=@]
module rec A : sig
  type t
  val f : t -> t
end =
struct
  type t = int
  let f (x : t) = A.f @\fbox{x}@
end
\end{lstlisting}
&
\begin{verbatim}
(* OCaml 4.02.03 fails with: *)
Error: This expression has type t = int
       but an expression was expected of type A.t
\end{verbatim}
\end{tabular}
\caption{Simple recursion in OCaml with type synonyms}
\label{fig:double-vision-simple-recursion-ocaml-synonym}
\end{figure}

\noindent
OCaml has determined the type of \verb|A.f| to be \verb|A.t -> A.t|,
while the boxed \verb|x| has type \verb|t|.  This example exhibits
``double vision'' because OCaml is unable to tell that \verb|A.t| and
\verb|t| are equal, despite the fact that, intuitively, both identifiers refer to
the same type.

A slight variation on the above litmus test is one which utilizes a generative type
definition (\verb|type t = MkT of int|) rather than a type synonym (\verb|type t = int|):

\begin{figure}[H]
\begin{tabular}{p{0.35\textwidth} p{0.55\textwidth}}
\begin{lstlisting}[language=ML,escapechar=@]
module rec A : sig
  type t
  val f : t -> t
end =
struct
  type t = @\colorbox{yellow!50}{MkT of int}@
  let f (x : t) = A.f @\fbox{x}@
end
\end{lstlisting}
&
\begin{verbatim}
(* OCaml 4.02.03 succeeds, with #show_module: *)
module A : sig type t val f : t -> t end
\end{verbatim}
\end{tabular}
\caption{Simple recursion in OCaml with generative data}
\label{fig:double-vision-simple-recursion-ocaml-generative}
\end{figure}

\noindent
Interestingly, OCaml is able to compile this example, because the type-checker
adds an equality \verb|t = A.t| when typechecking the structure; something that
is not possible when there is already a type equality \verb|t = int|.\footnote{\url{http://caml.inria.fr/pub/ml-archives/caml-list/2007/06/0d23465b5b04f72fedecdd3bbf2c9d72.en.html}}

\paragraph{Simple recursion in MixML}
Both RMC~\cite{dreyer:recursive} and MixML~\cite{rossberg+:mixml} are
both systems which solve the double vision problem.  As RMC was never
implemented, we ported our
litmus test to the MixML prototype interpreter 0.2.1, using the
desugaring of recursive modules described in the paper.

\begin{figure}[H]
\begin{tabular}{p{0.40\textwidth} p{0.50\textwidth}}
\begin{lstlisting}[language=ML,escapechar=@]
link A =
  {type t, val f : t -> t}
with {
  type t = int,
  val f (x : t) = A.f x
}
\end{lstlisting}
&
\begin{verbatim}
(* mixml 0.2.1 passes with type: *)
it : {f : [int -> int]+, t : [= int : #]}
\end{verbatim}
\end{tabular}
\caption{Simple recursion in MixML with type synonyms and sealing}
\label{fig:double-vision-simple-recursion-mixml}
\end{figure}

\noindent
An astute reader may note that the desugaring from the paper omits the
final sealing operation, unlike OCaml recursive modules which implicitly
seal by the forward declaration.  It is a simple matter to add the sealing back:

\begin{figure}[H]
\begin{tabular}{p{0.40\textwidth} p{0.50\textwidth}}
\begin{lstlisting}[language=ML,escapechar=@]
link A =
  {type t, val f : t -> t}
with ({
  type t = int,
  val f (x : t) = A.f x
} @\colorbox{yellow!50}{:> {type t, val f : t -> t}}@)
\end{lstlisting}
&
\begin{verbatim}
(* mixml 0.2.1 passes with type: *)
it.t%2 : #
it : {f : [it.t%2 -> it.t%2]+,
      t : [= it.t%2 : #]+}
\end{verbatim}
\end{tabular}
\caption{Simple recursion in MixML with type synonyms and sealing}
\label{fig:double-vision-simple-recursion-mixml-sealed}
\end{figure}

\noindent
MixML does not natively support generative data declarations, since they can be
encoded using modules and sealing.

\paragraph{Simple recursion in GHC Haskell} Recursive modules
with generative data can be directly ported to Haskell using \verb|hs-boot| files:

\begin{figure}[H]
\begin{tabular}{p{0.35\textwidth} p{0.55\textwidth}}
\begin{lstlisting}[language=Haskell,escapechar=@]
-- A.hs-boot
module A where
  data T
  f :: T -> T
-- B.hs
module B(f) where
  import {-# SOURCE #-} A
-- A.hs
module A(T, f) where
  import qualified B
  data T = MkT Int
  f :: T -> T
  f x = B.f x
\end{lstlisting}
&
\begin{verbatim}
{- GHC 8.0.2 succeeds -}
\end{verbatim}
\end{tabular}
\caption{Simple recursion in Haskell with \texttt{hs-boot} files}
\label{fig:double-vision-simple-recursion-haskell-hs-boot}
\end{figure}

\noindent
There is some nuance in the above translation, since GHC Haskell's module
system is substantially less expressive than ML's.  First, in most
presentations of recursive modules, the defined module is implicitly
\emph{sealed} by the signature that serves as the forward reference.
In GHC Haskell, no such sealing takes place: if a module directly
imports the implementing module, they will see all the entities it
exports.  Data abstraction in Haskell is achieved by \emph{hiding}
constructors of generative type definitions from the exports of a module, as we
have done here. If we were to implement \verb|T| using a type synonym
(not currently allowed by GHC), there is no way to seal it after-the-fact
so that it is opaque.

Second, when the abstract type declaration \verb|data T| is specified in
\verb|A.hs-boot| file, we are \emph{committing} to there being a
generative data declaration of \verb|T| in the implementation \verb|A|.
In particular, \verb|A| is not allowed to reexport a definition of
\verb|T| from somewhere else, nor are we permitted to implement \verb|T|
using a type synonym (as was the case in the first OCaml example.)
With these restrictions, it's quite easy to ``solve'' the double-vision
problem, since it easy to allocate a consistent original name for both
declarations of \verb|T| in \verb|A.hs-boot| and \verb|A.hs|.

\paragraph{Simple recursion in \OldBackpack{}}
\OldBackpack{} does not support implementing abstract types using type
synonyms, but does support implementing abstract types via a reexport.
Thus, it does need to address a variant of the double vision problem
on \emph{original names}.  Here is an example written in \OldBackpack{}
which constructed so as to not be typecheckable using GHC Haskell's existing strategy,
by creating a separate module \verb|M| defines \verb|T|
for \verb|A| to reexport.

\begin{figure}[H]
\begin{tabular}{p{0.35\textwidth} p{0.55\textwidth}}
\begin{lstlisting}[language=Haskell,escapechar=@]
module M where
    data T = MkT Int
signature A where
    data T
    f :: T -> T
module B(f) where
    import A
module A(T, x) where
    import qualified B
    import M (T)
    f :: T -> T
    f x = B.f x
\end{lstlisting}
&
\begin{verbatim}
{- Backpack'14 accepts -}
{- GHC 8.0.2 (with {-# SOURCE #-} added) fails: -}
A.hs-boot:2:1: error:
    'A.T' is exported by the hs-boot file,
    but not exported by the module

A.hs-boot:3:1: error:
    Identifier 'f' has conflicting definitions
    in the module and its hs-boot file
    Main module: f :: T -> T
    Boot file:   f :: A.T -> A.T
    The two types are different
\end{verbatim}
\end{tabular}
\caption{Simple recursion in \OldBackpack{}}
\label{fig:double-vision-simple-recursion-old-backpack}
\end{figure}

\noindent
In this case, GHC is clearly afflicted by the double vision
problem!\footnote{In fact, in GHC 8.0.2, if we were to refer to
\texttt{B.T}, that would cause a compiler panic.  The above example works
only ``accidentally'', because when we typecheck, \texttt{B.f} takes
on the type of the local definition of \texttt{f}, \emph{not} the
forward declaration.}

\section{Backwards propagating type information}

Both MixML and \OldBackpack{} avoid double-vision by way of a
type-computation/shaping pass, which computes the types/original names
of modules before typechecking proper.  This means that these
type systems permit type information can be propagated \emph{backwards}
from the relevant declaration.

Although this property increases the expressivity of a type
system, it interacts poorly with separate compilation, as the
pre-pass must be performed over the \emph{entire} recursive loop
before any typechecking can take place.

\iffalse%
\paragraph{Backwards propagation in MixML}  Consider the following
example:

\begin{figure}[H]
\begin{tabular}{p{0.40\textwidth} p{0.50\textwidth}}
\begin{lstlisting}[language=ML,escapechar=@]
{ type t,
  val f : t -> t,
  val x = f 2 }
    with
{ type t = int }
\end{lstlisting}
&
\begin{verbatim}
(* mixml 0.2.1 passes with type: *)
it : {f : [int -> int]-,
      t : [= int : #], x : [int]+}
\end{verbatim}
\end{tabular}
\caption{Backwards propagating type information in MixML}
\label{fig:double-vision-backwards-propagating-mixml}
\end{figure}

\noindent
In this example, a module which ordinarily would not typecheck
(there is no reason to believe \verb|t = int|) is mixin linked
with a second module, which introduces the necessary type equality
for \verb|f 2| to typecheck.  We think this is a bit \emph{unusual}.

\Red{But does MixML actually behave this way, or is it an
implementation bug? Point out that the suggested encoding for signature
refinement is not upheld faithfully here.}
\fi
\paragraph{Backwards propagation in \OldBackpack{}}

Consider the following adapation of
Figure~\ref{fig:double-vision-simple-recursion-old-backpack}, where
\verb|f MkT| has been moved out of \verb|module A| and into a module of
its own.

\begin{figure}[H]
\begin{tabular}{p{0.35\textwidth} p{0.55\textwidth}}
\begin{lstlisting}[language=Haskell,escapechar=@]
module M where
    data T = MkT
signature A where
    data T
    f :: T -> T
module B where
    import M
    import A
    g = f MkT
module A(T) where
    import B
    import M
\end{lstlisting}
&
\begin{verbatim}
{- Backpack'14 accepts -}
\end{verbatim}
\end{tabular}
\caption{Backwards propagation in \OldBackpack{}}
\label{fig:double-vision-backwards-propagating-old-backpack}
\end{figure}

\noindent
Why does this typecheck?  In \OldBackpack{}, a shaping pass ignores
all term declarations (including \verb|f MkT|) and computes the
original names of all types.  At this point, we learn that \verb|A.T|
is \verb|M.T|; now when \verb|B| is typechecked, \verb|f MkT| is found
to be well-typed.

This example gets at the heart of why \OldBackpack{} is not a very
separately-compileable design: we \emph{cannot} know if \verb|module B| is
type-correct without having first shaped \verb|module A|.

\paragraph{No backwards propagation in \Backpack{}}  When we originally
went about implementing Backpack, we implemented the shaping pass
as suggested in \OldBackpack{}.  But it quickly became evident that before
doing any typechecking, we would have to reshape all the modules in a package;
and for what?  Arguably, it is extremely surprising for the typechecking of
\verb|B| to depend on the source of the module that imports it!

Thus, we argue that type refinements should only be visible \emph{from the
module that implements the refinement and those which import it},
preserving separate compilation.  This means that shaping and type
computation can be done \emph{on-the-fly}, at the point when the refinement
occurs, we strengthen our context.  This might seem like a tall order,
but as we have seen \Red{reference}, GHC's type checker is up to the challenge.

\section{Mutual recursion}

\paragraph{Bidirectional Type Lookup}

In Section~\ref{subsec:typechecking-dependency}, we observed that
it is necessary to refine the specifications of types in signatures
before checking if a module matches a signature.  The bidirectional
type lookup problem arises when we are merging two signatures, in
which case type lookup and refinement must be performed in both
directions simultaneously.  Below, we've transcribed MixML's
example of bidirectional type lookup into Haskell:

%\begin{figure}[H]
\begin{tabular}{p{0.30\textwidth} p{0.30\textwidth} p{0.30\textwidth}}
\begin{verbatim}
signature A where
    type T = Int
    data U
    f :: Int -> U
\end{verbatim}
&
\begin{verbatim}
signature A where
    data T
    type U = Bool
    f :: T -> Bool
\end{verbatim}
&
\begin{verbatim}
signature A where
    type T = Int
    type U = Bool
    f :: Int -> U
\end{verbatim}
\end{tabular}
%\caption{Simple}
%\label{fig:signature-merging}
%\end{figure}

\noindent
Like MixML, \Backpack{} is able to handle the type refinement simply
by picking \verb|type T = Int| and \verb|type U = Bool| as the ``canonical''
representatives for the overall typechecking process, at which point
\verb|Int -> U| and \verb|T -> Bool| are definitionally equal.

An interesting question is whether or not \Backpack{} and MixML's semantics
diverge in any cases.  In this case, it is instructive to consider



This in and of itself
is not unusual: signature matching in plain ML operates similarly;
however, \Backpack{} goes about implementing this refinement quite differently
from ML or even MixML\@.  This 


\paragraph{Cyclic type definitions}

Like MixML, \Backpack{} must detect and reject type synonym cycles that
could arise after merging signatures:

\begin{tabular}{p{0.45\textwidth} p{0.45\textwidth}}
\begin{verbatim}
signature A where
    type T = U
    data U
\end{verbatim}
&
\begin{verbatim}
signature A where
    data T
    type U = T
\end{verbatim}
\end{tabular}

Also like MixML, \Backpack{} can merge signatures to form opaquely recursive types:

\begin{tabular}{p{0.45\textwidth} p{0.45\textwidth}}
\begin{verbatim}
signature A where
    data T = MkT U
    data U
\end{verbatim}
&
\begin{verbatim}
signature A where
    data T
    data U = MkU T
\end{verbatim}
\end{tabular}
