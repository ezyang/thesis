\chapter{Setting the stage}

\paragraph{What is Backpack trying to solve?} Backpack asks the
question, ``How can we add ML-like support for programming against
interfaces to Haskell?''  While one approach might be to directly
\emph{add} ML functors to Haskell as an extension, \Red{it wouldn't let
us write very many papers, and} this would be setting the bar too low:
ML functors are considered a ``heavy-weight'' language feature in no
small part due to a preponderance of sharing constraints.  Furthermore,
it is not possible to support mutually recursive linking using functors.
Backpack addressed these problems with an applicative, mix-in module
system which operated at the package level, rather than the module
level.

\paragraph{What are the problems with Backpack?}  While the original
Backpack paper argued that it was a practical design to retrofit on
Haskell, in fact, it did not adequately address many important,
non-obvious design constraints which we discovered in the course of implementing
Backpack.  In the following subsections, we spend some time
explaining implementation-induced constraints on the design of a
module system for Haskell.

\subsection{Respect the compilation / package manager boundary}

Traditionally, the way most users build programs in Haskell is not by
invoking the compiler (GHC) directly, but invoking a package manager
(Cabal).  The compiler is responsible for building individual modules,
while the package manager is responsible for building and installing
entire packages.  These two pieces of software are decoupled and
developed independently by different (only somewhat overlapping) teams.

As a \emph{package} system, Backpack must integrate not only with
the compiler, but also the package manager.  This is especially true
in an applicative module system, where components which have been
instantiated in the same way must be shared, even when they are compiled
as part of disparate projects: managing this deduplication is squarely
in the package manager's domain.

The original Backpack paper stylizes itself as a ``package language''
rather than a module language, but it fails to uphold this division
in its semantics: shaping is a process that should be carried out
by the package manager, but it requires fairly detailed knowledge
about Haskell source code.

\subsection{Optimizing against dependencies}
\label{sec:optimizing}

Today, the compiled interfaces of Haskell modules can record the source
code of an expression, to allow a user of this interface to
\emph{inline} the function in question.  This can help performance quite
a bit, especially when a module defines many small functions: inlining
exposes optimization opportunities.

A major goal we set is that modular development in Backpack should have
similar performance characteristics: you should not pay a performance
tax when you abstract some code over a signature.  A consequence of this
is that when we instantiate a package with mix-in linking, we need to
recompile it for the correct dependencies.  Compare this to Haskell's
type classes: functions using a type class are not specialized without
requesting inlining.

The primary implication of this goal is that it is far less important to
ensure that a package with some signatures can be directly linked
against its implementation; we can assume that when this linking occurs,
the compiler is going to re-compile (and re-typecheck!) the combined
entity.  The original Backpack design was completely oblivious to this fact.

\subsection{Instantiate packages, not modules}
\label{sec:instantiate-pkgs}

In the traditional Cabal model, dependencies declared for a package
affect all modules in the package, even if not all of the modules
actually (transitively) import them.  Thus, the dependency structure of
packages is manifestly clear without having to look at Haskell source
code, and a package can be compiled as a whole into a library file
(e.g., an \verb|so| or \verb|a| file), which then can be linked in
statically or dynamically by existing operating system facilities.
These properties should be preserved by Backpack.

The original Backpack design failed this property: modules had
their dependencies on signatures tracked individually, which would have
made it very difficult to actually compile a package into a single
shared library.\footnote{The situation gets even worse if one suggests
that each declaration should be considered a module.}  Furthermore, you
could not tell what the requirements of a module were without inspecting
the import graph.

\subsection{Incremental compilation}

When we make a change to Haskell source code, GHC only needs to recompile
any source code which may have (transitively) imported this source:
it can completely skip processing all other modules.  However, the
original Backpack paper defined a declaration shaping pass which had
to be computed on all modules before typechecking could proceed.  This
pass could not be easily incrementalized (as the act of merging shapes
could cause unification to occur) and was also completely unnecessary
in the vast majority of cases, as declaration shaping is really only
necessary to solve the double-vision problem which occurs during when
one has mutually recursive modules.\footnote{If one is willing to
specify ahead of time what the shape is, as is implicitly the case with GHC's
\texttt{hs-boot}, no shaping is necessary even with recursion.}

In the absence of mutual recursion, we can eliminate this extra shaping
pass and solve the recompilation problem in one fell swoop.

% It is simple to dodge this bullet, however: shaping is only necessary in
% the presence of mutual recursion.

\subsection{Migration}

There is already a large amount of code published in the Haskell
ecosystem, and it would be unreasonable to expect everyone to switch
over to Backpack overnight.  Dependencies specified by Backpack
signatures (indirect dependencies) must coexist with traditional
\verb|build-depends| dependencies (direct dependencies, modulo
dependency solving).  The original Backpack paper said nothing about
how to deal with this case.
